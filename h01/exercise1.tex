
\documentclass{abgabe}
\begin{document}

\begin{questions}
    \question
    Ein Tannenbaum soll zu Weihnachten bunte Lämpchen bekommen. 
    Dazu soll eine Lichterkette mit 5 gelben, 3 roten und 2 grünen Lämpchen zusammengestellt werden. 
    \begin{parts}
        \part 
        Wie viele Möglichkeiten gibt es, wenn es keine Einschränkungen gibt?
        \begin{solution}
            Dann gilt: (Permutation mit teilweise nicht unterscheidbaren Lampen)
            \[ 
                P(14;5;3;4;2) = \frac{14!}{5! \cdot 3! \cdot 4! \cdot 2!} = \num{2522520}
            \]
            \qed
        \end{solution}
        
        \part 
        Die 3 roten Lämpchen sollen nebeneinander angeordnet werden. 
        Wie viele Möglichkeiten gibt es? 
        \begin{solution}
            Dann gilt: (analog zu Aufgabenteil (a), wobei wir die roten Lampen als \gqq{eine Lampe} sehen)
            \[ 
                P(12;5;4;2) = \frac{12!}{5! \cdot 4! \cdot 2!} = \num{83160}
            \]
            \qed
        \end{solution}
        
        \part 
        Opa Hoppenstedt hat schon zwei gelbe Lampen an den Anfang der ette geschraubt und zwei gelbe Lampen an das Ende der Kette;
        er besteht darauf, diese nicht mehr zu verändern. 
        Wie viele Möglichkeiten der Anordnung ergeben sich für die restlichen Lampen? 
        \begin{solution}
            Dann gilt: (analog zu Aufgabenteil (a), jedoch mit nur noch einer gelben Lampe)
            \[ 
                P(10;3;4;2) = \frac{10!}{3! \cdot 4! \cdot 2!} = \num{12600}
            \]
            \qed
        \end{solution}
        
        \newpage
        \part 
        Jede Farbe wird als feste Kette geliefert, auf der die Lampen nicht mehr gewechselt werden können. 
        Wie viele Möglichkeiten gibt es, die 4 Ketten hintereinander zu schalten?
        \begin{solution}
            Dann gilt: (Permutation)
            \[ 
                P(4) = 4! = 24
            \]
            \qed
        \end{solution}
    \end{parts}
\end{questions}
\end{document}