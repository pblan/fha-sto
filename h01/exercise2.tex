
\documentclass{abgabe}
\begin{document}

\begin{questions}
    \question
    Im Portemonnaie befinden sich fünf 50 Cent Stücke und sieben 5 Cent Stücke. 
    Drei Münzen werden zufällig und ohne Zurücklegen entnommen. 
    Wie viel Geld hat man
    \begin{parts}
        \part 
        mindestens auf der Hand?
        \begin{solution}
             \[ 
                3 \cdot 5 \ \text{Cent} = 15 \ \text{Cent}
             \] 
             \qed
        \end{solution}
        
        \part 
        höchstens auf der Hand?
        \begin{solution}
            \[ 
                3 \cdot 50 \ \text{Cent} = 150 \ \text{Cent}
             \] 
            \qed
        \end{solution}
        
        \part 
        Wie viele Kombinationen von Münzen (nicht Werten) können auftreten?
        \begin{solution}
            Durch die Bemerkung gehen wir davon aus, dass die Münzen an sich auch bei gleichem Wert unterscheidbar bleiben.

             Dann gilt: (Kombinationen ohne Wiederholung)
             \[ 
                C(12;3) = \binom{12}{3} = 220
             \] 
            \qed
        \end{solution}
        
        \part 
        Wie oft kommt dabei der Betrag von $0.6$ Euro zustande?
        \begin{solution}
            Wir wählen uns eine beliebige 50 Cent Münze und zwei beliebige 5 Cent Münzen. 

            Dann gilt: (jeweils Kombinationen ohne Wiederholung)
            \[ 
                n = C(5;1) \cdot C(7;2) = 5 \cdot 21 = 105
            \] 
            \qed
        \end{solution}
        
    \end{parts}
\end{questions}
\end{document}