
\documentclass{abgabe}
\begin{document}

\begin{questions}
    \question
    In der Formel 1 gehen je nach Saison 10 Teams mit jeweils 2 Formel 1-Fahrern an den Start.
    \begin{parts}
        \part 
        Wie viele Möglichkeiten gibt es, die Formel 1-Wagen an der Startlinie aufzustellen?
        \begin{solution}
            Insgesamt fahren nach Aufgabenstellung (ich habe nämlich keine Ahnung von Formel 1) 20 Wagen pro Rennen mit. 

            Damit gilt: (Permutation)
            \[ 
                P(20) = 20! = \num{2432902008176640000}
            \] 
            \qed

            Alternativ könnte man annehmen, dass die zwei Wagen der jeweiligen Hersteller nicht unterscheidbar sind, da sie - nach Wikipedia - baugleich sind. 

            Dann gilt: (Permutation mit 10x 2 identischen Wagen)
            \[ 
            P(20;2;2;2;2;2;2;2;2;2;2) = \frac{20!}{\left(2!\right)^{10}} = \num{2375880867360000}
            \] 
            \qed
        \end{solution}
        
        \part 
        Mit Ihren Formel 1-FreundInnen schauen Sie sich jedes Rennen gemeinsam an. 
        Im Vorfeld überlegen Sie, welche Formel 1-Fahrer die ersten drei Plätze belegen. 
        Wie viele Möglichkeiten hierfür gibt es? 
        \begin{solution}
            Damit gilt: (Permutationen)
            \[ 
            n = \frac{20!}{17!} = 20 \cdot 19 \cdot 18 = \num{6840}
            \] 
            \qed
        \end{solution}
    \end{parts}
\end{questions}
\end{document}