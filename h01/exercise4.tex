
\documentclass{abgabe}
\begin{document}

\begin{questions}
    \question
    Ein Fußballmannschaft besteht bekanntlich aus 11 Spielern.
    \begin{parts}
        \part 
        Die 11 Spieler verlassen vor Spielbeginn der Reihe nach die Mannschaftskabine. 
        Wie viele verschiedene Reihenfolgen sind dabei möglich?
        \begin{solution}
            Es gilt: (Permutation) 
            \[ 
                P(11) = 11! = \num{39916800}
            \] 
            \qed
        \end{solution}
        
        \part 
        Der Trainer will für ein Elfmeterschießen 5 Spieler aus seiner Mannschaft auswählen. 
        Wie viele Möglichkeiten hierfür gibt es?
        \begin{solution}
            Es gilt: (Kombination ohne Wiederholung)
            \[ 
                C(11;5) = \binom{11}{5} = 462
            \] 
            \qed
        \end{solution}

        \part 
        Der Trainer entscheidet sich dafür, 5 Spieler der Mannschaft für das Elfmeterschießen auszuwählen und gleichzeitig die Reihenfolge festzulegen, in welcher die 5 Spieler zum Elfmeter antreten sollen. 
        Wie viele Möglichkeiten gibt es für dieses Auswahlverfahren?
        \begin{solution}
            Nach Aufgabenteil (b) haben wir 462 Möglichkeiten um die 5 Spieler auszuwählen. 

            Dann gilt: (Multipliziert mit Permutationsanzahl)
            \[ 
                n = C(11;5) \cdot P(5) = 462 \cdot 120 = \num{55400}
            \] 
            \qed
        \end{solution}
    \end{parts}
\end{questions}
\end{document}