
\documentclass{abgabe}
\begin{document}

\begin{questions}
    \question
    Eine homogene Münze wird \emph{dreimal} geworfen (\gqq{Zahl}: $Z$, \gqq{Wappen}: $W$).
    \begin{parts}
        \part
        Bestimmen Sie die dabei möglichen Ergebnisse (Elementarereignisse), sowie die Ergebnismenge $\Omega$ dieses Zufallsexperiments.
        \begin{solution}
            Es gilt:
            \[
                \begin{aligned}
                    \Omega =
                    \{
                     & (Z,Z,Z), (Z,Z,W), (Z,W,Z), (Z,W,W), \\
                     & (W,Z,Z), (W,Z,W), (W,W,Z), (W,W,W)
                    \}
                \end{aligned}
            \]
            \qed
        \end{solution}

        \part
        Durch welche Teilmengen von $\Omega$ lassen sich die folgenden Ereignisse beschreiben?
        \begin{subparts}
            \subpart $A := \{ \text{Bei drei Würfen zweimal \gqq{Zahl}} \}$
            \begin{solution}
                Wir gehen davon aus, dass \gqq{mindestens zweimal} gemeint ist.
                Es gilt:
                \[
                    \begin{aligned}
                        A =
                        \{
                         & (Z,Z,Z), (Z,Z,W), (Z,W,Z), (W,Z,Z)
                        \}
                    \end{aligned}
                \]
                \qed
            \end{solution}

            \subpart $B := \{ \text{Bei drei Würfen zweimal \gqq{Wappen}} \}$
            \begin{solution}
                Wir gehen davon aus, dass \gqq{mindestens zweimal} gemeint ist.
                Es gilt:
                \[
                    \begin{aligned}
                        B =
                        \{
                         & (Z, W, W), (W,Z,W), (W,W,Z), (W, W, W)
                        \}
                    \end{aligned}
                \]
                \qed
            \end{solution}

            \newpage
            \subpart $D := \{ \text{Bei drei Würfen dreimal \gqq{Zahl}} \}$
            \begin{solution}
                Es gilt:
                \[
                    \begin{aligned}
                        C =
                        \{
                         & (Z,Z,Z)
                        \}
                    \end{aligned}
                \]
                \qed
            \end{solution}

            \subpart $E := \{ \text{Bei drei Würfen dreimal \gqq{Wappen}} \}$
            \begin{solution}
                Es gilt:
                \[
                    \begin{aligned}
                        D =
                        \{
                         & (W,W,W)
                        \}
                    \end{aligned}
                \]
                \qed
            \end{solution}
        \end{subparts}

        \part
        Bilden Sie aus den unter (b) genannten Ereignissen die folgenden zusammengesetzten Ereignisse und deuten Sie diese:
        \begin{subparts}
            \subpart
            $A \cup B$
            \begin{solution}
                Es gilt:
                \[
                    \begin{aligned}
                        A \cup B =
                        \{
                         & (Z,Z,Z), (Z,Z,W), (Z,W,Z), (Z,W,W), \\
                         & (W,Z,Z), (W,Z,W), (W,W,Z), (W,W,W)
                        \}
                    \end{aligned}
                \]
                Interpretation:
                \[
                    A \cup B = \{ \text{Eine homogene Münze wird dreimal geworfen} \}
                \]
                \qed
            \end{solution}

            \subpart
            $B \cup E$
            \begin{solution}
                Es gilt:
                \[
                    \begin{aligned}
                        B \cup E =
                        \{
                         & (Z, W, W), (W,Z,W), (W,W,Z), (W,W,W)
                        \}
                    \end{aligned}
                \]
                Interpretation:
                \[
                    B \cup E = \{ \text{Bei drei Würfen wird höchstens einmal Zahl geworfen} \}
                \]
                \qed
            \end{solution}

            \newpage
            \subpart
            $D \cup E$
            \begin{solution}
                Es gilt:
                \[
                    \begin{aligned}
                        D \cup E =
                        \{
                         & (Z,Z,Z), (W,W,W)
                        \}
                    \end{aligned}
                \]
                Interpretation:
                \[
                    D \cup E = \{ \text{Bei drei Würfen wird dreimal das gleiche geworfen} \}
                \]
                \qed
            \end{solution}

            \subpart
            $A \cap B$
            \begin{solution}
                Es gilt:
                \[
                    \begin{aligned}
                        A \cap B =
                        \emptyset
                    \end{aligned}
                \]
                Interpretation:
                \[
                    A \cap B = \{ \text{Bei drei Würfen wird zweimal Kopf und zweimal Wappen geworfen} \}
                \]
                \qed
            \end{solution}
        \end{subparts}
    \end{parts}
\end{questions}
\end{document}