
\documentclass{abgabe}

\begin{document}

\begin{questions}
    \question
    Ein Land hat 14 Millionen Einwohner.
    Davon sind

    \begin{center}
        \num{700000} Einwohner arbeitslos \\
        \num{400000} Einwohner Akademiker \\
        \num{100000} Einwohner arbeitslose Akademiker
    \end{center}

    Bestimmen Sie die Wahrscheinlichkeit, dass
    \begin{parts}
        \part
        ein beliebiger Einwohner arbeitslos ist.
        \begin{solution}
            Seien $A := \{ \text{Ein Einwohner ist arbeitslos} \}$ und $B := \{ \text{Ein Einwohner ist Akademiker} \}$.

            Wir wissen:
            \[
                P(A) = \frac{\num{700000}}{\num{14000000}} = \frac{1}{20}
            \]
            \qed
        \end{solution}

        \part
        ein beliebiger Einwohner ein Akademiker ist.
        \begin{solution}
            Wir wissen:
            \[
                P(B) = \frac{\num{400000}}{\num{14000000}} = \frac{1}{35}
            \]
            \qed
            %P(A \cap B) & = \frac{\num{100000}}{\num{14000000}} = \frac{1}{140}
        \end{solution}

        \part
        ein Einwohner arbeitslos ist, wenn man weiß, dass er ein Akademiker ist.
        \begin{solution}
            Es gilt:
            \[
                P(A \mid B) = \frac{P(A \cap B)}{P(B)} = \frac{\frac{\num{100000}}{\num{14000000}}}{\frac{1}{35}} = \frac{35}{140} = \frac{1}{4}
            \]
            \qed
        \end{solution}

        \newpage
        \part
        ein Einwohner Akademiker ist, wenn man weiß, dass er ein Arbeitsloser ist?
        \begin{solution}
            Es gilt:
            \[
                P(B \mid A) = \frac{P(A \cap B)}{P(A)} = \frac{20}{140} = \frac{1}{7}
            \]
            \qed
        \end{solution}
    \end{parts}
\end{questions}
\end{document}