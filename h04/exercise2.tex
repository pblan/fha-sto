
\documentclass{abgabe}

\begin{document}

\begin{questions}
    \question
    Gegeben seien die folgenden, jeweils auf $\R$ definierten Funktionen:
    \begin{parts}
        \part
        $F_1(x) =
            \begin{cases}
                0   & \text{für} \ x < 2        \\
                x-2 & \text{für} \ 2 \leq x < 4 \\
                1   & \text{für} \ x \geq 4
            \end{cases}
        $
        \begin{solution}
            Für $x_1 = \nicefrac{7}{2}$ und $x_2 = 2$ gilt:
            \[
                F_1(x_1) = \frac{3}{2} > 1 = F_2(x_2) \quad \land \quad x_1 < x_2 \quad \lightning
            \]

            Also ist $F_1(x)$ nicht monoton steigend und damit insgesamt keine Verteilungsfunktion.
            \qed
        \end{solution}

        \part
        $F_2(x) =
            \begin{cases}
                0      & \text{für} \ x < 0    \\
                e^{-x} & \text{für} \ x \geq 0
            \end{cases}
        $
        \begin{solution}
            Offensichtlich ist $F_2(x)$ (streng) monoton fallend für $x \geq 0$ und damit insgesamt keine Verteilungsfunktion.
            \qed
        \end{solution}

        \part
        $F_3(x) = e^{-e^{-x}}$ für $x \in \R$
        \begin{solution}
            Offensichtlich ist $F_3(x)$ monoton steigend und rechtsseitig stetig.

            Es gilt:
            \[
                \lim_{x \to \infty} F_3(x) = \lim_{x \to \infty} e^{-e^{-x}} = e^{- \lim_{x \to \infty} e^{-x}} = e^{0} = 1
            \]
            \[
                \lim_{x \to -\infty} F_3(x) = \lim_{x \to -\infty} e^{-e^{-x}} = e^{- \lim_{x \to -\infty} e^{-x}} = e^{-\infty} = 0
            \]

            Also ist $F_3(x)$ damit insgesamt eine Verteilungsfunktion.
            \qed
        \end{solution}
    \end{parts}
    Welche dieser Funktionen können nicht Verteilungsfunktionen einer Zufallsvariablen sein?
    Begründen Sie Ihre Antwort.
\end{questions}
\end{document}