
\documentclass{abgabe}

\begin{document}

\begin{questions}
    \question
    Gegeben sei die diskrete Zufallsvariable $X$.
    Betrachten Sie folgende zugehörige Wahrscheinlichkeitsfunktion:
    \[
        f(x) =
        \begin{cases}
            \frac{x(5-x)}{20} & \text{für} \ x = \{ 1,2,3,4 \} \\
            0                 & \text{sonst}
        \end{cases}
    \]
    \begin{parts}
        \part
        Zeichnen Sie die Wahrscheinlichkeitsfunktion $f(x)$.
        \begin{solution}
            \begin{center}
                \includegraphics[width=.7\linewidth]{exercise3_bar_chart}
            \end{center}
        \end{solution}

        \part
        Berechnen Sie die Verteilungsfunktion $F(x)$.
        \begin{solution}
            Die Wahrscheinlichkeitsverteilung der \emph{diskreten} Zufallsvariablen $X$ lässt sich durch die \emph{Verteilungsfunktion}
            \[
                P(X \leq x) = F(x) =
                \begin{cases}
                    0               & \text{für} \ x < 1        \\
                    \nicefrac{1}{5} & \text{für} \ 1 \leq x < 2 \\
                    \nicefrac{1}{2} & \text{für} \ 2 \leq x < 3 \\
                    \nicefrac{4}{5} & \text{für} \ 3 \leq x < 4 \\
                    1               & \text{für} \ x \geq 4
                \end{cases}
            \]
            beschreiben.
            \qed
        \end{solution}

        \newpage
        \part
        Stellen Sie diese Verteilungsfunktion grafisch dar.
        \begin{solution}
            \begin{center}
                \includegraphics[width=.7\linewidth]{exercise3_cfd}
            \end{center}
        \end{solution}
    \end{parts}
\end{questions}
\end{document}