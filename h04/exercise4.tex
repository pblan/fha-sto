
\documentclass{abgabe}

\begin{document}

\begin{questions}
    \question
    Die Verspätung eines Zuges in einem bestimmten Bahnhof werde durch die stetige Zufallsvariable $X$ beschrieben und habe die Dichtefunktion (in Minuten)
    \[
        f(x) =
        \begin{cases}
            \frac{1}{2} - \frac{1}{8}x & \text{für} \ 0 \leq x \leq 4 \\
            0                          & \text{sonst}
        \end{cases}
    \]
    \begin{parts}
        \part
        Erfüllt die angegebene Funktion $f(x)$ die Anforderung an eine Dichtefunktion?
        \begin{solution}
            Es muss gelten:
            \begin{itemize}
                \item $f$ ist nichtnegativ \quad $\checkmark$ (offensichtlich)
                \item $f$ ist integrierbar \quad $\checkmark$ (offensichtlich)
                \item $f$ ist normiert:
                      \[
                          \int^{\infty}_{-\infty} f(t) \diff t = \int^{4}_{0} \left( \frac{1}{2} - \frac{1}{8}x \right) \diff t = \left[\frac{x}{2} - \frac{x^2}{16}\right]^{4}_{0} = 2 - 1 = 1 \quad \checkmark
                      \]
            \end{itemize}

            Damit ist $f(x)$ eine Dichtefunktion.
            \qed
        \end{solution}

        \part
        Geben Sie die Verteilungsfunktion von $X$ an.
        \begin{solution}
            Es gilt:
            \[
                P(X \leq x) = \int_{-\infty}^{x} f(t) \diff t = \int_{-\infty}^{x} \left(\frac{1}{2} - \frac{1}{8}t\right) \diff t = \int_{0}^{x} \left(\frac{1}{2} - \frac{1}{8}t\right) \diff t = \frac{x}{2} - \frac{x^2}{16}
            \]

            Und damit:
            \[
                P(X \leq x) = F(x) =
                \begin{cases}
                    0                                    & \text{für} \ x < 0           \\
                    \nicefrac{x}{2} - \nicefrac{x^2}{16} & \text{für} \ 0 \leq x \leq 4 \\
                    1                                    & \text{sonst}
                \end{cases}
            \]
            \qed
        \end{solution}

        \newpage
        \part
        Sie haben bereits eine Minute auf den Zug gewartet.
        Wie groß ist die Wahrscheinlichkeit, dass die heutige Verspätung zwischen zwei und drei Minuten beträgt?
        \begin{solution}
            Offensichtlich gilt:
            \begin{alignat*}{1}
                P(2 \leq X \leq 3 \mid X \geq 1) & = \frac{P(2 \leq X \leq 3 \cap X \geq 1)}{P(X \geq 1)} \\
                                                 & = \frac{P(2 \leq X \leq 3)}{P(X \geq 1)}               \\
                                                 & = \frac{F(3) - F(2)}{1 - F(1)}                         \\
                                                 & = \frac{\frac{15}{16} - \frac{3}{4}}{1 - \frac{7}{16}} \\
                                                 & = \frac{15 - 12}{9}                                    \\
                                                 & = \frac{1}{3}
            \end{alignat*}
            \qed
        \end{solution}
    \end{parts}
\end{questions}
\end{document}