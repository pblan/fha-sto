
\documentclass{abgabe}

\begin{document}

\begin{questions}
    \question
    Es sei $X$ stetig mit Dichtefunktion $f: (0;\infty) \to (0;\infty); f(x) = \gamma x e^{-x}$ mit $x > 0$.
    \begin{parts}
        \part
        Bestimmen Sie $\gamma$ so, dass $f$ eine Dichtefunktion ist.
        \begin{solution}
            Damit $f$ eine Dichtefunktion ist, muss gelten:
            \begin{alignat*}{2}
                                                       & \int_0^\infty f(x) \diff x                                        &  & = 1                \\
                \equiv \quad                           & \int_0^\infty \gamma x e^{-x} \diff x                             &  & = 1                \\
                \equiv \quad                           & \gamma \int_0^\infty x e^{-x} \diff x                             &  & = 1                \\
                \equiv \quad                           & \left[ -xe^{-x} \right]_0^\infty + \int_0^\infty e^{-x} \diff x   &  & = \frac{1}{\gamma} \\
                \equiv \quad                           & \left[ -xe^{-x} \right]_0^\infty - \left[ e^{-x} \right]_0^\infty &  & = \frac{1}{\gamma} \\
                \equiv \quad                           & -\lim_{x\to\infty} xe^{-x} - \left( 0 - 1 \right)                 &  & = \frac{1}{\gamma} \\
                \stackrel{\footnotemark}{\equiv} \quad & 0  + 1                                                            &  & = \frac{1}{\gamma} \\
                \equiv \quad                           & \gamma                                                            &  & = 1
            \end{alignat*}
            \qed
        \end{solution}
        \footnotetext{$e^x$ wächst asymptotisch schneller als $x$.}

        \newpage
        \part
        Wie lautet die Verteilungsfunktion von $X$?
        \begin{solution}
            Es gilt (mit $x > 0$):
            \[
                P(X \leq x) = F(x)
                = \int_{0}^{x} f(t) \diff t
                = \int_{0}^{x} t e^{-t} \diff t
                \stackrel{\footnotemark}{=}  - \left[ (1+t)e^{-t} \right]_{0}^{x}
                = 1-(1+x)e^{-x}
            \]
            \qed
        \end{solution}
        \footnotetext{Berechnung analog zu Teilaufgabe (a)}

        \part
        Bestimmen Sie die Verteilung von $Y = \frac{1}{X}$
        \begin{solution}
            Es gilt:
            \[
                F_Y(y) = P(Y \leq y) = P \left(\frac{1}{X} \leq y\right) = P \left(X \geq \frac{1}{y}\right) = 1 - P \left(X < \frac{1}{y}\right) = 1 - F_X \left(\frac{1}{y}\right)
            \]

            Und damit:
            \[
                F_Y(y) = 1 - F_X\left(\frac{1}{y}\right) = 1 - \left( 1 -\left( 1+\frac{1}{y} \right)e^{-\frac{1}{y}} \right) = \left( 1+\frac{1}{y} \right)e^{-\frac{1}{y}}
            \]
            \qed
        \end{solution}
    \end{parts}
\end{questions}
\end{document}