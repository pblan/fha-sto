
\documentclass{abgabe}

\begin{document}

\begin{questions}
    \question
    $X$ repräsentiere die täglichen Verkäufe eines bestimmten Produktes und besitze die Wahrscheinlichkeitsverteilung:
    \begin{center}
        \begin{tabular}{|C|C|C|C|C|C|C|C|C|}
            \hline
            x_i        & \num{7000} & \num{7500} & \num{8000} & \num{8500} & \num{9000} & \num{9500} & \num{10000} \\
            \hline
            P(X = x_i) & 0.05       & 0.2        & 0.35       & 0.19       & 0.12       & 0.08       & 0.01        \\
            \hline
        \end{tabular}
    \end{center}
    \begin{parts}
        \part
        Berechnen Sie
        \begin{subparts}
            \subpart
            den Erwartungswert,
            \begin{solution}
                Es gilt:
                \begin{alignat*}{1}
                    \Mean(X) & = \sum^{6}_{i=0} x_i \cdot P(X = x_i)                                                          \\
                             & = \num{7000} \cdot 0.05 + \num{7500} \cdot 0.2 + \num{8000} \cdot 0.35 + \num{8500} \cdot 0.19 \\
                             & \phantom{= } + \num{9000} \cdot 0.12 + \num{9500} \cdot 0.08 + \num{10000} \cdot 0.01          \\
                             & = 8205
                \end{alignat*}
                \qed
            \end{solution}

            \subpart
            die Varianz
            \begin{solution}
                Es gilt:
                \begin{alignat*}{1}
                    \Var(X) & = \sum^{6}_{i=0} (x_i - \mu_X)^2 \cdot P(X = x_i)                                                       \\
                            & = (\num{7000} - 8205)^2 \cdot 0.05 + (\num{7500} - 8205)^2 \cdot 0.2 + (\num{8000} - 8205)^2 \cdot 0.35 \\
                            & \phantom{= } + (\num{8500} - 8205)^2 \cdot 0.19 + (\num{9000} - 8205)^2 \cdot 0.12                      \\
                            & \phantom{= } + (\num{9500} - 8205)^2 \cdot 0.08 + (\num{10000} - 8205)^2 \cdot 0.01                     \\
                            & = \num{445475}
                \end{alignat*}
                \qed
            \end{solution}

            \subpart
            und den Median von $X$
            \begin{solution}
                Es gilt:
                \[
                    P(X < \tilde{x}) \leq \frac{1}{2} \quad \land \quad P(X \leq \tilde{x}) \geq \frac{1}{2}
                \]

                Offensichtlich gilt nach der gegebenen Tabelle:
                \[
                    \tilde{x} = 8000
                \]
                \qed
            \end{solution}

        \end{subparts}

        \part
        Berechnen Sie das
        \begin{subparts}
            \subpart
            untere Quartil,
            \begin{solution}
                Es gilt:
                \[
                    P(X < x_{\nicefrac{1}{4}}) \leq \frac{1}{4} \quad \land \quad P(X \leq x_{\nicefrac{1}{4}}) \geq \frac{1}{4}
                \]

                Offensichtlich gilt nach der gegebenen Tabelle:
                \[
                    x_{\nicefrac{1}{4}} = 7500
                \]
                \qed
            \end{solution}

            \subpart
            obere Quartil
            \begin{solution}
                Es gilt:
                \[
                    P(X < x_{\nicefrac{3}{4}}) \leq \frac{3}{4} \quad \land \quad P(X \leq x_{\nicefrac{3}{4}}) \geq \frac{3}{4}
                \]

                Offensichtlich gilt nach der gegebenen Tabelle:
                \[
                    x_{\nicefrac{3}{4}} = 8500
                \]
                \qed
            \end{solution}

            \subpart
            sowie den Quartilsabstand
            \begin{solution}
                Es gilt:
                \[
                    x_{\nicefrac{3}{4}} - x_{\nicefrac{1}{4}} = 8500 - 7500 = 1000
                \]
                \qed
            \end{solution}

        \end{subparts}

        \part
        Berechnen Sie das $90\%$-Quantil.
        \begin{solution}
            Es gilt:
            \[
                P(X < x_{\nicefrac{9}{10}}) \leq \frac{9}{10} \quad \land \quad P(X \leq x_{\nicefrac{9}{10}}) \geq \frac{9}{10}
            \]

            Offensichtlich gilt nach der gegebenen Tabelle:
            \[
                x_{\nicefrac{9}{10}} = 9000
            \]
        \end{solution}
    \end{parts}
\end{questions}
\end{document}