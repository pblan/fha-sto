
\documentclass{abgabe}

\begin{document}
\begin{questions}
    \question
    Ein Unternehmen hat einen neuen Auftrag erhalten.
    Die zu produzierenden Werkstücke sollen eine bestimmte Länge haben.
    Der Kunde akzeptiert eine Toleranz von $\pm 0.5\si{\mm}$.
    Aus Erfahrung weiß man im Unternehmen, dass die Wahrscheinlichkeit für die Abweichungen von Sollgrößen (gemessen in \si{\mm}) mit folgender Dichtefunktion beschrieben werden kann:
    \[
        f(x) =
        \begin{cases}
            0.25(3+x) & \text{für} \ -3 \leq x < 0   \\
            0.25(3-x) & \text{für} \ 0 \leq x \leq 3 \\
            0         & \text{sonst}
        \end{cases}
    \]
    \begin{parts}
        \part
        Wie groß ist die Wahrscheinlichkeit, ein Werkstück zu liefern, das vom Kunden auch angenommen wird?
        \begin{solution}
            Die Dichtefunktion ist nicht normiert, weswegen die Aufgabe nicht sinnvoll lösbar ist.
        \end{solution}

        \part
        Wie groß ist das Moment 1. Ordnung (= Erwartungswert)?
        \begin{solution}
            Die Dichtefunktion ist nicht normiert, weswegen die Aufgabe nicht sinnvoll lösbar ist.
        \end{solution}

        \part
        Wie groß ist das Zentralmoment 2. Ordnung (= Varianz)?
        \begin{solution}
            Die Dichtefunktion ist nicht normiert, weswegen die Aufgabe nicht sinnvoll lösbar ist.
        \end{solution}
    \end{parts}
\end{questions}
\end{document}