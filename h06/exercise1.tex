
\documentclass{abgabe}

\begin{document}
\begin{questions}
    \question
    Die Wahrscheinlichkeit, dass die Zündung bei einem Auto falsch eingestellt ist, sei $p = 0.3$.
    Es werden $n=5$ Autos ausgewählt.
    Die betrachtete Zufallsvariable $X$ bezeichnet die Zahl der Autos mit falsch eingestellter Zündung.
    \begin{parts}
        \part
        Bestimmen Sie für $X = 0,1,2,3,4,5$ die Werte
        \begin{subparts}
            \subpart
            der Wahrscheinlichkeits- und
            \begin{solution}
                Wir wissen, dass es sich um eine \emph{Binomialverteilung} handelt mit
                \begin{itemize}
                    \item $n = 5$,
                    \item $p = 0.3$.
                \end{itemize}
                Damit gilt für $ x \in \interval{0,5}_{\N_0}$:
                \[
                    b(x;n,p) = b(x;5,0.3) = f(x) = P(X=x) = \binom{n}{x} \cdot p^x \cdot q^{n-x} = \binom{5}{x} \cdot \left( \frac{3}{10} \right)^x \cdot \left( \frac{7}{10} \right)^{5-x}
                \]

                Und damit:

                \begin{center}
                    \begin{tabular}{C||C|C|C|C|C|C}
                        x    & 0        & 1        & 2       & 3       & 4       & 5       \\
                        \hline
                        f(x) & 16.807\% & 36.015\% & 30.87\% & 13.23\% & 2.835\% & 0.243\%
                    \end{tabular}
                \end{center}
                \qed
            \end{solution}

            \subpart
            der Verteilungsfunktion.
            \begin{solution}
                Es gilt für $x \in \interval{0,5}_{\N_0}$:
                \[
                    B(x;n,p) = B(x;5,0.3) = F(x) = P(X \leq x) = \sum_{k \leq x} b(x;n,p) = \sum_{k \leq x} \binom{5}{x} \cdot \left( \frac{3}{10} \right)^x \cdot \left( \frac{7}{10} \right)^{5-x}
                \]

                Und damit:

                \begin{center}
                    \begin{tabular}{C||C|C|C|C|C|C}
                        x    & 0        & 1        & 2        & 3        & 4        & 5     \\
                        \hline
                        F(x) & 16.807\% & 52.822\% & 83.692\% & 96.922\% & 99.757\% & 100\%
                    \end{tabular}
                \end{center}
                \qed
            \end{solution}
        \end{subparts}

        \newpage
        \part
        Berechnen Sie die Wahrscheinlichkeit, dass
        \begin{subparts}
            \subpart
            bei 2 Autos die Zündung falsch eingestellt ist.
            \begin{solution}
                Offensichtlich gilt nach Teilaufgabe (a):
                \[
                    f(2) = 30.87\%
                \]
                \qed
            \end{solution}

            \subpart
            bei 2 oder weniger Autos die Zündung falsch eingestellt ist.
            \begin{solution}
                Offensichtlich gilt nach Teilaufgabe (a):
                \[
                    F(2) = 83.692\%
                \]
                \qed
            \end{solution}

            \subpart
            bei mehr als 3 Autos die Zündung falsch eingestellt ist.
            \begin{solution}
                Offensichtlich gilt:
                \[
                    P(X > 3) = 1 - P(X < 3) = 1 - P(X \leq 2) = 1 - F(2) = 1 - 83.692\% = 16.308\%
                \]
                \qed
            \end{solution}
        \end{subparts}

        \part
        Berechnen Sie den Erwartungswert und die Varianz.
        \begin{solution}
            Es gilt:
            \[
                \mu = np = 5 \cdot 0.3 = 1.5 \quad \land \quad \sigma^2 = npq = 5 \cdot 0.3 \cdot 0.7 = 1.05
            \]
            \qed
        \end{solution}
    \end{parts}
\end{questions}
\end{document}