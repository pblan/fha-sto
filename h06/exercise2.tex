
\documentclass{abgabe}

\begin{document}
\begin{questions}
    \question
    Unter 50 Glühbirnen in einem Karton befinden sich 5 Defekte.
    Bei einer Qualitätskontrolle werden 3 Birnen getestet.
    Wie groß ist die Wahrscheinlichkeit, dass
    \begin{parts}
        \part
        alle 3 defekt sind,
        \begin{solution}
            Wir wissen, dass es sich um eine \emph{hypergeometrische} Verteilung handelt mit
            \begin{itemize}
                \item $N = 50$,
                \item $M = 5$,
                \item $n = 3$.
            \end{itemize}

            Damit gilt:
            \[
                h(x;N,M,n) = h(x;50,5,3) = f(x) = P(X=x) = \frac{\binom{M}{x} \binom{N-M}{n-x}}{\binom{N}{n}} = \frac{\binom{5}{x} \binom{45}{3-x}}{\binom{50}{3}}
            \]

            Und damit:
            \[
                P(X = 3) = h(3;50,5,3) = f(3) = \frac{1}{1960} \approx 0.05102\%
            \]
            \qed
        \end{solution}

        \part
        genau 2 defekt sind,
        \begin{solution}
            Es gilt:
            \[
                P(X = 2) = h(2;50,5,3) = f(2) = \frac{9}{392} \approx 2.29592\%
            \]
            \qed
        \end{solution}

        \part
        zwischen einer und drei Birnen defekt sind,
        \begin{solution}
            Es gilt:
            \[
                P(1 \leq X \leq 3) = \sum^3_{k=1} P(X = k) = \sum^3_{k=1} h(k;50,5,3) = \frac{541}{1960} \approx 27.602\%
            \]
            \qed
        \end{solution}

        \part
        keine defekt ist.
        \begin{solution}
            Es gilt:
            \[
                P(X = 0) = h(0;50,5,3) = \frac{1419}{1960} \approx 72.39796\%
            \]
            \qed
        \end{solution}

        \newpage
        \part
        Wie viele defekte Birnen sind bei dieser Stichprobe im Mittel zu erwarten?
        \begin{solution}
            Es gilt:
            \[
                \mu = n \cdot \frac{M}{N} = 3 \cdot \frac{5}{50} = \frac{3}{10}
            \]
            \qed
        \end{solution}
    \end{parts}
\end{questions}
\end{document}