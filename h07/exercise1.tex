
\documentclass{abgabe}
\begin{document}

\begin{questions}
    \question
    Die Qualität der Kugeln für Kugellager wird auf folgende Weise kontrolliert: 
    Fällt die Kugel durch eine Öffnung mit dem Durchmesser $d_2$, jedoch nicht durch eine Öffnung mit dem Durchmesser $d_1$ ($d_1 < d_2$), so genügt die Kugel den Qualitätsanforderungen.
    Wird eine der beiden Bedingungen nicht eingehalten, ist die Kugel Ausschuss.
    \begin{parts}
        \part 
        Es ist bekannt, dass der Durchmesser $D$ der Kugeln unter den gegebenen Fertigungsbedingungen eine normalverteilte zufällige Größe mit den Parametern
        \[ 
            \mu = \frac{d_1 + d_2}{2} \quad \text{und} \quad \sigma = \frac{d_2 - d_1}{4}
        \]
        ist.
        Bestimmen Sie die Ausschussquote $p$, d.h. die Wahrscheinlichkeit dafür, dass eine beliebige Kugel sich als Ausschuss erweist.
        \begin{solution}
            Offensichtlich gilt: 
            \[ 
                D \sim N(\mu, \sigma^2) = N\left(\frac{d_1 + d_2}{2}, \left( \frac{d_2 - d_1}{4}\right)^2 \right)
            \]
            und damit für die Wahrscheinlichkeit, dass die Kugel genügt: 
            \begin{alignat*}{1}
                P(d_1 \leq D \leq d_2) & = \Phi \left( \frac{d_2 - \mu}{\sigma} \right) - \Phi \left( \frac{d_1 - \mu}{\sigma} \right)                                                           \\
                                       & = \Phi \left( \frac{d_2 - \frac{d_1 + d_2}{2}}{\frac{d_2 - d_1}{4}} \right) - \Phi \left( \frac{d_1 - \frac{d_1 + d_2}{2}}{\frac{d_2 - d_1}{4}} \right) \\
                                       & = \Phi \left( 4 \cdot \frac{d_2 - \frac{d_1 + d_2}{2}}{d_2 - d_1} \right) - \Phi \left( 4 \cdot \frac{d_1 - \frac{d_1 + d_2}{2}}{d_2 - d_1} \right)     \\
                                       & = \Phi \left( 4 \cdot \frac{2d_2 - d_1 - d_2}{2(d_2 - d_1)} \right) - \Phi \left( 4 \cdot \frac{d_1 - d_2}{2(d_2 - d_1)} \right)                        \\
                                       & = \Phi \left( 2 \cdot \frac{d_2 - d_1}{d_2 - d_1} \right) - \Phi \left( 2 \cdot \frac{d_1 - d_2}{d_2 - d_1} \right)                                     \\
                                       & = \Phi \left( 2 \right) - \Phi \left( 2 \cdot \frac{-(d_2 - d_1)}{d_2 - d_1} \right)                                                                    \\
                                       & = \Phi \left( 2 \right) - \Phi \left( -2 \right)                                                                                                        \\
                                       & = \Phi \left( 2 \right) - (1 - \Phi \left( 2 \right))                                                                                                   \\
                                       & = 2 \cdot \Phi \left( 2 \right) - 1                                                                                                                     \\
                                       & = 2 \cdot 0.9772  - 1                                                                                                                                   \\
                                       & = 1.9544  - 1                                                                                                                                           \\
                                       & = 95.44\%                                                                                                                            
            \end{alignat*}
            
            Damit ist also die Ausschussquote $p = 1 - P(d_1 \leq D \leq d_2) = 4.56\%$.
            \qed
        \end{solution}
        
        \newpage
        \part 
        Es ist bekannt, dass der Durchmesser $D$ normalverteilt mit $\mu = \nicefrac{d_1+d_2}{2}$ ist und dass der Ausschuss 10\% der gesamten Partie ausmacht. 
        Bestimmen Sie $\sigma$.
        \begin{solution}
            Es gilt: 
            \begin{alignat*}{2}
                               & P(d_1 \leq D \leq d_2) &  & = \Phi \left( \frac{d_2 - \mu}{\sigma} \right) - \Phi \left( \frac{d_1 - \mu}{\sigma} \right)                             \\
                \equiv \quad   & 0.9                    &  & = \Phi \left( \frac{d_2 - \frac{d_1+d_2}{2}}{\sigma} \right) - \Phi \left( \frac{d_1 - \frac{d_1+d_2}{2}}{\sigma} \right) \\
                \equiv \quad   & 0.9                    &  & = \Phi \left( \frac{d_2 - d_1}{2\sigma} \right) - \Phi \left( \frac{d_1 - d_2}{2\sigma} \right)                           \\
                \equiv \quad   & 0.9                    &  & = \Phi \left( \frac{d_2 - d_1}{2\sigma} \right) - \Phi \left( - \frac{d_2 - d_1}{2\sigma} \right)                         \\
                \equiv \quad   & 0.9                    &  & = \Phi \left( \frac{d_2 - d_1}{2\sigma} \right) - \left( 1 - \Phi \left( \frac{d_2 - d_1}{2\sigma} \right) \right)        \\
                \equiv \quad   & 1.9                    &  & = 2\cdot \Phi \left( \frac{d_2 - d_1}{2\sigma} \right)                                                                    \\
                \equiv \quad   & 0.95                   &  & = \Phi \left( \frac{d_2 - d_1}{2\sigma} \right)                                                                           \\
                \implies \quad & \Phi (1.65)            &  & \approx \Phi \left( \frac{d_2 - d_1}{2\sigma} \right)                                                                     \\
                \equiv \quad   & 1.65                   &  & \approx \frac{d_2 - d_1}{2\sigma}                                                                                         \\
                \equiv \quad   & 2\sigma                &  & \approx \frac{d_2 - d_1}{1.65}                                                                                            \\ 
                \equiv \quad   & \sigma                 &  & \approx \frac{d_2 - d_1}{3.3}
            \end{alignat*}
            \qed
        \end{solution}
    \end{parts}
\end{questions}
\end{document}