
\documentclass{abgabe}
\begin{document}

\begin{questions}
    \question
    Die Qualität der Kugeln für Kugellager wird auf folgende Weise kontrolliert: 
    Fällt die Kugel durch eine Öffnung mit dem Durchmesser $d_2$, jedoch nicht durch eine Öffnung mit dem Durchmesser $d_1$ ($d_1 < d_2$), so genügt die Kugel den Qualitätsanforderungen.
    Wird eine der beiden Bedingungen nicht eingehalten, ist die Kugel Ausschuss.
    \begin{parts}
        \part 
        Es ist bekannt, dass der Durchmesser $D$ der Kugeln unter den gegebenen Fertigungsbedingungen eine normalverteilte zufällige Größe mit den Parametern
        \[ 
            \mu = \frac{d_1 + d_2}{2} \quad \text{und} \quad \sigma = \frac{d_2 - d_1}{4}
        \]
        ist.
        Bestimmen Sie die Ausschussquote $p$, d.h. die Wahrscheinlichkeit dafür, dass eine beliebige Kugel sich als Ausschuss erweist.
        \begin{solution}

            \qed
        \end{solution}
        
        \part 
        Es ist bekannt, dass der Durchmesser $D$ normalverteilt mit $\mu = \nicefrac{d_1+d_2}{2}$ ist und dass der Ausschuss 10\% der gesamten Partie ausmacht. 
        Bestimmen Sie $\sigma$.
        \begin{solution}

            \qed
        \end{solution}
    \end{parts}
\end{questions}
\end{document}