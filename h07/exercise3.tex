
\documentclass{abgabe}
\begin{document}

\begin{questions}
    \question
    Die Lebensdauer (in Stunden) von Energiesparlampen eines bestimmten Fabrikats kann durch eine mit dem Parameter $\lambda > 0$ expoentialverteilte Zufallsvariable $X$ beschrieben werden. 
    Die zugehörige Verteilungsfunktion $F : R \to [0; 1]$ ist damit gegeben durch:
    \[ 
        F(x) = 
        \begin{cases}
            0                  & x < 0    \\ 
            1 - e^{-\lambda x} & x \geq 0
        \end{cases}
    \]
    \begin{parts}
        \part 
        Berechnen Sie für $\lambda = \nicefrac{1}{800}$ die Wahrscheinlichkeit dafür, dass die Lebensdauer einer derartigen Energiesparlampe
        \begin{subparts}
            \subpart 
            höchstens 300 Stunden,
            \begin{solution}

                \qed
            \end{solution}
            
            \subpart 
            mehr als 120 Stunden,
            \begin{solution}

                \qed
            \end{solution}
            
            \subpart 
            mindestens 240 und höchstens 360 Stunden
            \begin{solution}

                \qed
            \end{solution}
        \end{subparts}
        beträgt.
        
        \part 
        Für welchen Wert des Parameters $\lambda$ ergibt sich eine Lebensdauerverteilung, bei der mit Wahrscheinlichkeit $0.99$ die Lebensdauer einer derartigen Energiesparlampe mindestens 100 Stunden beträgt?
        \begin{solution}

            \qed
        \end{solution}
    \end{parts}
\end{questions}
\end{document}