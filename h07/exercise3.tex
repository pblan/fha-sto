
\documentclass{abgabe}
\begin{document}

\begin{questions}
    \question
    Die Lebensdauer (in Stunden) von Energiesparlampen eines bestimmten Fabrikats kann durch eine mit dem Parameter $\lambda > 0$ exponentialverteilte Zufallsvariable $X$ beschrieben werden. 
    Die zugehörige Verteilungsfunktion $F : R \to [0; 1]$ ist damit gegeben durch:
    \[ 
        F(x) = 
        \begin{cases}
            0                  & x < 0    \\ 
            1 - e^{-\lambda x} & x \geq 0
        \end{cases}
    \]
    \begin{parts}
        \part 
        Berechnen Sie für $\lambda = \nicefrac{1}{800}$ die Wahrscheinlichkeit dafür, dass die Lebensdauer einer derartigen Energiesparlampe
        \begin{subparts}
            \subpart 
            höchstens 300 Stunden,
            \begin{solution}
                Offensichtlich gilt: 
                \[ 
                    P(X \leq 300) = F(300) = 1 - e^{-\nicefrac{300}{800}} = 1 - e^{-\nicefrac{3}{8}}
                \]
                \qed
            \end{solution}
            
            \subpart 
            mehr als 120 Stunden,
            \begin{solution}
                Offensichtlich gilt: 
                \[ 
                    P(X \geq 120) = 1 - F(120) = 1 - \left( 1 - e^{-\nicefrac{120}{800}} \right) = e^{-\nicefrac{3}{20}}
                \]
                \qed
            \end{solution}
            
            \subpart 
            mindestens 240 und höchstens 360 Stunden
            \begin{solution}
                Offensichtlich gilt: 
                \[ 
                    P(240 \leq X \leq 360) = F(360) - F(240) = 1 - e^{-\nicefrac{360}{800}} - \left( 1 - e^{-\nicefrac{240}{800}} \right) = e^{-\nicefrac{3}{10}} - e^{-\nicefrac{9}{20}}
                \]
                \qed
            \end{solution}
        \end{subparts}
        beträgt.
        
        \newpage 
        \part 
        Für welchen Wert des Parameters $\lambda$ ergibt sich eine Lebensdauerverteilung, bei der mit Wahrscheinlichkeit $0.99$ die Lebensdauer einer derartigen Energiesparlampe mindestens 100 Stunden beträgt?
        \begin{solution}
            Es gilt:
            \[ 
                0.99 = P(X \geq 100) = 1 - P(X < 100) = 1 - F(100) \iff F(100) = 0.01
            \]
            \begin{alignat*}{2}
                               & F(100)              &  & = 0.01                  \\
                \equiv \quad   & 1 - e^{-100\lambda} &  & = 0.01                  \\
                \equiv \quad   & e^{-100\lambda}     &  & = 0.100                 \\
                \equiv \quad   & -100\lambda         &  & = \ln 0.100             \\
                \equiv \quad   & \lambda             &  & = \frac{\ln 0.99}{-100} \\
                \implies \quad & \lambda             &  & \approx 0.0001005
            \end{alignat*}
            \qed
        \end{solution}
    \end{parts}
\end{questions}
\end{document}