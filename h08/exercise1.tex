
\documentclass{abgabe}
\begin{document}

\begin{questions}
    \question
    In einem Beutel befinden sich 6 Münzen: eine 5-Cent-Münze, drei 2-Cent-Münzen und zwei 1-Cent-Münzen. 
    Zufällig werden nacheinander - ohne Zurücklegen - 2 Münzen gezogen. 
    $X_1$ gebe den Wert der ersten, $X_2$ den Wert der zweiten gezogenen Münzen an. 
    Bestimmen Sie folgenden Werte:
    \begin{parts}
        \part 
        die zweidimensionale Wahrscheinlichkeitsfunktion $P(X_1 = i, X_2 = j)$ für $i,j \in \{1; 2; 5\}$.
        \begin{solution}
            Offensichtlich gilt: 
            \begin{center}
                \begin{tabular}{|C|C|C|C|C|}
                    \hline 
                    TO DO & 1   & 2    & 5    & f_1(X_1) \\ % \diagbox{X_1}{X_2}
                    \hline 
                    1                & \nicefrac{1}{15} & \nicefrac{1}{5} & \nicefrac{1}{15} & \nicefrac{1}{3} \\
                    \hline 
                    2                & \nicefrac{1}{5} & \nicefrac{1}{5} & \nicefrac{1}{10} & \nicefrac{1}{2} \\
                    \hline 
                    5                & \nicefrac{1}{15} & \nicefrac{1}{10} & 0 & \nicefrac{1}{6} \\
                    \hline 
                    f_2(X_2)                 & \nicefrac{1}{3}    & \nicefrac{1}{2}     & \nicefrac{1}{6}     & 1 \\
                    \hline
                \end{tabular}
            \end{center}
            \qed
        \end{solution}
        
        \part 
        den Erwartungswert $\Mean(X_i)$ und die Varianz $\Var(X_i)$ ($i = 1,2$).
        \begin{solution}
            Für $X_1$ gilt offensichtlich: 
            \[ 
            \mu_{X_1} = \Mean(X_1) = \sum_{i} i \cdot f_1(i) = \ldots = \frac{13}{6} \approx 2.1667
            \] 
            \[ 
            \sigma^2_{X_1} = \Var(X_1) = E((X_1 - \mu_{X_1}^2)) = \sum_{i} \left( i \cdot f_1(i) - \mu_{X_1} \right)^2 = \ldots = \frac{13}{2} = 6.5
            \] 

            In der Wahrscheinlichkeitsfunktion ist zu sehen, dass die Randverteilungen für beide Zufallsvariablen gleich sind. 
            Damit gilt offensichtlich:
            \[ 
                \mu_{X_2} = \Mean(X_2) = \mu_{X_1} = \frac{13}{6} \approx 2.1667 \quad \land \quad \sigma^2_{X_2} = \Var(X_2) = \sigma^2_{X_1} = \frac{13}{2} = 6.5
            \] 
            \qed
        \end{solution}
        
        \newpage
        \part 
        die Kovarianz $\Cov(X_1,X_2)$ sowie den Korrelationskoeffizient $\rho_{X_1,X_2}$.
        \begin{solution}
            Es gilt mit $i,j \in \{1,2,5\}$: 
            \begin{alignat*}{1}
            \Cov(X_1, X_2) & = \Mean(X_1X_2) - \Mean(X_1)\Mean(X_2) = \sum_i \sum_j i \cdot j \cdot f(i,j) - \Mean(X_1)\Mean(X_2) \\ 
            & = \left( \frac{1}{15} + \frac{2}{5} + \frac{1}{3} + \frac{2}{5} + \frac{4}{5} + 1 + \frac{1}{3} + 1 + 0 \right) - \frac{13}{6} \cdot \frac{13}{6} \\ 
            & = \frac{13}{3} - \frac{169}{36}  = -\frac{13}{36} \approx -0.3611
            \end{alignat*}

            und 

            \[ 
                \rho_{X_1X_2} = \frac{\Cov(X,Y)}{\sigma_{X_1} \cdot \sigma_{X_2}} = \frac{-\frac{13}{36}}{\sqrt{\frac{13}{2}} \cdot \sqrt{\frac{13}{2}}} = -\frac{1}{18} \approx -0.0556
            \] 
            \qed
        \end{solution}
    \end{parts}
\end{questions}
\end{document}