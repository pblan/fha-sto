
\documentclass{abgabe}
\begin{document}

\begin{questions}
    \question
    Bei einem Produktionsvorgang werden Zylinder in den ausgefrästen Kreis eines Metallsockels eingepasst. 
    Die beiden Teile werden rein zufällig aus den bisher produzierten Zylindern bzw. ausgefrästen Metallplatten ausgewählt. 
    Der Durchmesser des Zylinders ist (in \si{\mm}) nach $N(24,9; (0,03)2)$-verteilt, der Durchmesser des in den Metallsockel eingefrästen Kreises ist nach $N(25; (0.04)2)$-verteilt. 
    Der Zylinder kann noch eingepasst werden, falls die lichte Weite der Durchmessers (= Durchmesser des gefrästen Kreises - Durchmesser des Zylinders) nicht mehr als $0.2\si{\mm}$ beträgt.
    \begin{parts}
        \part 
        Berechnen Sie 
        \begin{subparts}
            \subpart
            den Erwartungswert
            \begin{solution}

                \qed
            \end{solution}
        
            \subpart
            die Varianz 
            \begin{solution}

                \qed
            \end{solution}
            
            \subpart
            die Verteilung
            \begin{solution}

                \qed
            \end{solution}
        \end{subparts}
        der Zufallsvariablen \enquote{lichte Weite des Durchmessers}.
        
        \part 
        In wie viel Prozent aller Fälle lässt sich der Zylinder nicht in die Metallplatte einpassen?
        \begin{solution}

            \qed
        \end{solution}
    \end{parts}
\end{questions}
\end{document}