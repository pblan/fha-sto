\documentclass{abgabe}
\begin{document}

\begin{questions}
    \question
    In einer Urne befinden sich 4 Kugeln. 
    Zwei tragen die Aufschrift 1, die beiden anderen dagegen die Aufschrift 2 und 6. 
    Anton zieht (ohne Zurücklegen) zwei der Kugeln und erhält die Differenz als Gewinn in Euro ausgezahlt.
    \begin{parts}
        \part
        Welchen Einsatz sollte Anton zahlen, damit das Spiel fair ist?
        \begin{solution}

            \qed
        \end{solution}
    \end{parts}
    Anton muss im Folgenden pro Spiel 3€ Einsatz zahlen.
    \begin{parts}
        \setcounter{partno}{1}
        \part
        Welchen durchschnittlichen Reingewinn erwartet Anton jetzt pro Spiel? 
        Berechnen Sie auch die Varianz und die Standardabweichung des Reingewinns
        \begin{solution}

            \qed
        \end{solution}
        
        \part
        Anton spielt das Spiel insgesamt 90 Mal.
        Berechnen Sie für den Gesamtreingewinn den Erwartungswert und die Varianz.
        \begin{solution}

            \qed
        \end{solution}
        
        \part
        Schätzen Sie mit der Ungleichung von Tschebyscheff die Wahrscheinlichkeit dafür ab, dass der \enquote{Gesamtreingewinn} um mindestens 30€ vom Erwartungswert abweicht.
        \begin{solution}

            \qed
        \end{solution}
    \end{parts}
\end{questions}
\end{document}