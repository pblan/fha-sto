\documentclass{abgabe}
\begin{document}

\begin{questions}
    \question
    In einer Urne befinden sich 4 Kugeln. 
    Zwei tragen die Aufschrift 1, die beiden anderen dagegen die Aufschrift 2 und 6. 
    Anton zieht (ohne Zurücklegen) zwei der Kugeln und erhält die Differenz als Gewinn in Euro ausgezahlt.
    \begin{parts}
        \part
        Welchen Einsatz sollte Anton zahlen, damit das Spiel fair ist?
        \begin{solution}
            Wir nutzen erst einmal eine Nebenrechnung:
            
            Seien $X_1$ und $X_2$ Zufallsvariablen mit 
            \[ 
                X_1 := \{ \text{Aufschrift der ersten Kugel} \}
            \]
            \[ 
                X_2 := \{ \text{Aufschrift der zweiten Kugel} \}
            \]
            Dann gilt:
            
            \begin{center}
                \begin{tabular}{|C|C|C|C|C|}
                    \hline
                    $\diagbox{$X_1$}{$X_2$}$ & 1               & 2                & 6                & f_2(X_2)        \\ 
                    \hline
                    1                        & \nicefrac{1}{6} & \nicefrac{1}{6}  & \nicefrac{1}{6}  & \nicefrac{1}{2} \\
                    \hline
                    2                        & \nicefrac{1}{6} & 0                & \nicefrac{1}{12} & \nicefrac{1}{4} \\
                    \hline
                    6                        & \nicefrac{1}{6} & \nicefrac{1}{12} & 0                & \nicefrac{1}{4} \\
                    \hline
                    f_1(X_1)                 & \nicefrac{1}{2} & \nicefrac{1}{4}  & \nicefrac{1}{4}  & 1               \\
                    \hline
                \end{tabular}
            \end{center}
            
            Mit dieser Nebenrechnung gilt dann: 
            \[ 
                X = \{ \text{Gewinn in Euro ohne Einsatz} \}
            \]
            
            mit: 
            \begin{center}
                \begin{tabular}{|C|C|C|C|C|}
                    \hline
                    x    & 0               & 1               & 4               & 5               \\ 
                    \hline 
                    f(x) & \nicefrac{1}{6} & \nicefrac{1}{3} & \nicefrac{1}{6} & \nicefrac{1}{3} \\
                    \hline
                \end{tabular}
            \end{center}
            
            und damit: 
            \[ 
                \mu_X = \Mean(X) = 1 \cdot \frac{1}{3} + 4 \cdot \frac{1}{6} + 5 \cdot \frac{1}{3} = \frac{8}{3}
            \]
            
            Das Spiel ist also fair bei einem Einsatz von $\nicefrac{8}{3}$ Euro.
            \qed
        \end{solution}
    \end{parts}
    \newpage
    Anton muss im Folgenden pro Spiel 3€ Einsatz zahlen.
    \begin{parts}
        \setcounter{partno}{1}
        \part
        Welchen durchschnittlichen Reingewinn erwartet Anton jetzt pro Spiel? 
        Berechnen Sie auch die Varianz und die Standardabweichung des Reingewinns
        \begin{solution}
            Es gilt: 
            \[ 
                Y := \{ \text{Reingewinn in Euro mit Einsatz} \} \quad \iff \quad Y = X - 3
            \]
            
            Und damit: 
            \[ 
                \mu_Y = \Mean(Y) = \Mean(X) - 3 = -\frac{1}{3}
            \]
            \[
                \sigma_Y^2 = \Var(Y) = \Var(X) = \Mean((X - \mu_X)^2) = \sum_{i=1}^4 (x_i - \mu_X)^2 \cdot f(x_i) = \ldots = \frac{38}{9} \approx 4.22
            \]
            \[
                \sigma_Y = \sqrt{\sigma_Y^2} = \sqrt{\frac{38}{9}} = \frac{\sqrt{38}}{3} \approx 2.0548    
            \]
            \qed
        \end{solution}
        
        \part
        Anton spielt das Spiel insgesamt 90 Mal.
        Berechnen Sie für den Gesamtreingewinn den Erwartungswert und die Varianz.
        \begin{solution}
            Sei 
            \[ 
                Z = Y_1 + \ldots + Y_{90}
            \]
            Dann gilt: 
            \[ 
                \mu_Z = \Mean(Z) = 90 \cdot E(Y) = -30
            \]
            \[
                \sigma_Z^2 = \Var(Z) = 90 \cdot \Var(Y) = 90 \cdot \frac{38}{9} = 380
            \]
            \qed
        \end{solution}
        
        \part
        Schätzen Sie mit der Ungleichung von Tschebyscheff die Wahrscheinlichkeit dafür ab, dass der \enquote{Gesamtreingewinn} um mindestens 30€ vom Erwartungswert abweicht.
        \begin{solution}
            Es gilt offensichtlich mit Tschebyscheff: 
            \begin{alignat*}{2}
                             & P\left( \abs{Z - \mu_Z} \geq \epsilon \right) &  & \leq \frac{\Var(Z)}{\epsilon^2}                      \\ 
                \equiv \quad & P\left( \abs{Z + 30} \geq 30 \right)          &  & \leq \frac{380}{30^2}                                \\ 
                \equiv \quad & P\left( \abs{Z + 30} \geq 30 \right)          &  & \leq \frac{380}{900} = \frac{19}{45} \approx 42.22\%
            \end{alignat*}
            \qed
        \end{solution}
    \end{parts}
\end{questions}
\end{document}