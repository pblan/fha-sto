
\documentclass{abgabe}
\begin{document}

\begin{questions}
    \question
    Im letzten Wintersemester nahmen 120 Studierende an der Stochastik-Klausur teil.
    Im folgenden ist Punkteverteilung einer Stochastik-Klausur-Aufgabe angegeben:
    \begin{parts}
        \part
        Berechnen Sie
        \begin{subparts}
            \subpart
            den Erwartungswert.
            \begin{solution}

                \qed
            \end{solution}

            \subpart
            die Varianz und die Standardabweichung.
            \begin{solution}

                \qed
            \end{solution}
        \end{subparts}

        \part
        Mit welcher Wahrscheinlichkeit liegt der Punkteschnitt im Bereich $\interval{\mu-2, \mu + 2}$?
        Nutzen Sie zur Abschätzung die Tschebyscheff-Ungleichung.
        \begin{solution}

            \qed
        \end{solution}
    \end{parts}
\end{questions}
\end{document}