
\documentclass{abgabe}
\begin{document}

\begin{questions}
    \question
    Im letzten Wintersemester nahmen 120 Studierende an der Stochastik-Klausur teil.
    Im folgenden ist Punkteverteilung einer Stochastik-Klausur-Aufgabe angegeben:
    \begin{center}
        \begin{tabular}{|c|C|C|C|C|C|C|}
            \hline
            Punkte $X$   & 6 & 5                  & 4                  & 3                  & 2                  & 1                  \\
            \hline
            Anzahl       & 0 & 10                 & 30                 & 40                 & 20                 & 20                 \\
            \hline
            $P(X = x_i)$ & 0 & \nicefrac{10}{120} & \nicefrac{30}{120} & \nicefrac{40}{120} & \nicefrac{20}{120} & \nicefrac{20}{120} \\
            \hline
        \end{tabular}
    \end{center}
    \begin{parts}
        \part
        Berechnen Sie
        \begin{subparts}
            \subpart
            den Erwartungswert.
            \begin{solution}
                Es gilt offensichtlich:
                \[
                    \mu = \Mean(X) = \sum_{i=1}^6 x_i \cdot P(X = x_i) = \ldots = \frac{35}{12} \approx 2.9167
                \]
                \qed
            \end{solution}

            \subpart
            die Varianz und die Standardabweichung.
            \begin{solution}
                Es gilt offensichtlich:
                \[
                    \sigma^2 = \Var(X) = \sum_{i=1}^6 (x_i - \mu)^2 \cdot P(X = x_i) = \ldots = \frac{203}{144} \approx 1.4097
                \]
                \[
                    \sigma = \sqrt{\sigma^2} = \sqrt{\frac{203}{144}} = \frac{\sqrt{203}}{12} \approx 1.187
                \]
                \qed
            \end{solution}
        \end{subparts}

        \part
        Mit welcher Wahrscheinlichkeit liegt der Punkteschnitt im Bereich $\interval{\mu-2, \mu + 2}$?
        Nutzen Sie zur Abschätzung die Tschebyscheff-Ungleichung.
        \begin{solution}
            Es gilt offensichtlich mit Tschebyscheff:
            \begin{alignat*}{2}
                             & P\left( \abs{X - \mu} \geq \epsilon \right)     &  & \leq \frac{\Var(X)}{\epsilon^2}         \\
                \equiv \quad & 1 - P\left( \abs{X - \mu} < \epsilon \right)    &  & \leq \frac{\Var(X)}{\epsilon^2}         \\
                \equiv \quad & P\left( \abs{X - \mu} < \epsilon \right)        &  & \geq 1 - \frac{\Var(X)}{\epsilon^2}     \\
                \equiv \quad & P\left( \abs{X - \nicefrac{35}{12}} < 2 \right) &  & \geq 1 - \frac{203}{144 \cdot 2^2}      \\
                \equiv \quad & P\left( \abs{X - \nicefrac{35}{12}} < 2 \right) &  & \geq 1 - \frac{203}{576} \approx 0.6476
            \end{alignat*}
            \qed
        \end{solution}
    \end{parts}
\end{questions}
\end{document}