
\documentclass{abgabe}
\begin{document}

\begin{questions}
    \question
    Das Abwassersystem einer Gemeinde, an das 1332 Haushalte angeschlossen sind, ist für eine maximale Last von \num{13500} Litern pro Stunde ausgelegt.

    Nehmen Sie an, dass die einzelnen Abwassermengen (pro Stunde) von $n$ angeschlossenen Haushalten beschrieben werden können durch stochastisch unabhängige Zufallsvariablen $X_1, \ldots , X_n$, wobei $X_i$ für alle $i \in \{1, \ldots ,n\}$ normalverteilt ist mit Erwartungswert $\mu = 10$ (Liter/Stunde) und Varianz $\sigma^2 = 4$ ($\text{(Liter/Stunde)}^2$).
    Berechnen Sie
    \begin{parts}
        \part
        den Erwartungswert und die Varianz für die 1332 angeschlossenen Haushalte.
        \begin{solution}
            Offensichtlich gilt mit:
            \[
                X = X_1 + \ldots + X_{1332}
            \]
            direkt auch:
            \[
                \mu_X = 1332 \cdot \mu = \num{13320} \si{\L\per\hour} \quad \land \quad \sigma_X^2 = 1332 \cdot \sigma^2 = \num{5328} \si{\square\L\per\square\hour}
            \]
            \qed
        \end{solution}

        \part
        die Wahrscheinlichkeit einer Überlastung des Abwassersystems (für 1332 angeschlossene Haushalte).
        \begin{solution}
            Offensichtlich gilt auch:
            \[
                X \sim N(\mu,\sigma^2) = N(13320, \num{5328})
            \]

            Sei
            \[
                u = \frac{x - \mu}{\sigma} = \frac{\num{13500} - 13320}{\sqrt{\num{5328}}} = \frac{13500 - 13320}{12\sqrt{37}} = \frac{15\sqrt{37}}{37}
            \]

            Dann gilt:
            \[
                P(X > u) = 1 - P(X \leq u) = 1 - \Phi\left(\frac{15\sqrt{37}}{37}\right) \approx 1 - \Phi\left(2.47\right) = 1 - 0.9932 = 0.68\%
            \]
            \qed
        \end{solution}
    \end{parts}
\end{questions}
\end{document}