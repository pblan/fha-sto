
\documentclass{abgabe}
\begin{document}

\begin{questions}
    \question
    Das Abwassersystem einer Gemeinde, an das 1.332 Haushalte angeschlossen sind, ist für eine maximale Last von 13500 Litern pro Stunde ausgelegt.

    Nehmen Sie an, dass die einzelnen Abwassermengen (pro Stunde) von $n$ angeschlossenen Haushalten beschrieben werden können durch stochastisch unabhängige Zufallsvariablen $X_1, \ldots , X_n$, wobei $X_i \forall i \in \{1, \ldots ,n\}$ normalverteilt ist mit Erwartungswert $\mu = 10$ (Liter/ Stunde) und Varianz $\sigma^2 = 4$ ((Liter/ Stunde)2).
    Berechnen Sie
    \begin{parts}
        \part
        den Erwartungswert und die Varianz für die 1332 angeschlossenen Haushalte.
        \begin{solution}

            \qed
        \end{solution}

        \part
        die Wahrscheinlichkeit einer Überlastung des Abwassersystems (für 1332 angeschlossene Haushalte).
        \begin{solution}

            \qed
        \end{solution}
    \end{parts}
\end{questions}
\end{document}