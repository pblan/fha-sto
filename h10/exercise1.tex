\documentclass{abgabe}
\begin{document}

\begin{questions}
    \question
    In den 30 Museen der Stadt Artima gab es im letzen Monat jeweils $X$ Neuerwerbungen pro Museum. 
    Dabei sei folgende Urliste entstanden:
    \begin{center}
        \begin{tabular}{cccccccccc}
            2 & 4 & 3 & 5 & 5 & 2 & 3 & 1 & 5 & 6  \\ 
            4 & 7 & 8 & 3 & 2 & 8 & 3 & 6 & 4 & 6  \\ 
            5 & 7 & 3 & 3 & 2 & 5 & 4 & 4 & 3 & 11 \\
        \end{tabular}
    \end{center}
    \begin{parts}
        \part
        Erstellen Sie eine Tabelle mit der absoluten und relativen Häufigkeit bzw. Summenhäufigkeit der Neuerwerbungen $X$ pro Museum.
        \begin{solution}
            Es gilt: 
            \begin{center}
                \begin{tabular}{|C|C|C|C|}
                    \hline 
                    x_i & n_i & h_i              & H_i               \\ 
                    \hline 
                    1   & 1   & \nicefrac{1}{30} & \nicefrac{1}{30}  \\
                    2   & 4   & \nicefrac{2}{15} & \nicefrac{1}{6}   \\
                    3   & 7   & \nicefrac{7}{30} & \nicefrac{2}{5}   \\
                    4   & 5   & \nicefrac{1}{6}  & \nicefrac{17}{30} \\
                    5   & 5   & \nicefrac{1}{6}  & \nicefrac{11}{15} \\
                    6   & 3   & \nicefrac{1}{10} & \nicefrac{5}{6}   \\
                    7   & 2   & \nicefrac{1}{15} & \nicefrac{9}{10}  \\
                    8   & 2   & \nicefrac{1}{15} & \nicefrac{29}{30} \\
                    11  & 1   & \nicefrac{1}{30} & 1                 \\ 
                    \hline 
                \end{tabular}
            \end{center}
            \qed
        \end{solution}
        
        \newpage
        \part
        Zeichnen Sie im Anschluss
        \begin{subparts}
            \subpart 
            das zugehörige Stabdiagramm     
            \begin{solution}
                \begin{center}
                    \includegraphics[width=.7\linewidth]{exercise1_bar_chart}
                \end{center}
            \end{solution}
            
            \newpage
            \subpart 
            die empirische Verteilungsfunktion.   
            \begin{solution}
                \begin{center}
                    \includegraphics[width=.7\linewidth]{exercise1_cfd}
                \end{center}
            \end{solution}
        \end{subparts}
        
        \part
        Berechnen Sie
        \begin{subparts}
            \subpart 
            das arithmetische Mittel
            \begin{solution}
                Es gilt: 
                \[ 
                    \conj{x} = \frac{1}{n} \cdot \sum_i x_i = \frac{1}{30} \cdot 134 = \frac{67}{15} \approx 4.467
                \]
                \qed
            \end{solution}
            
            \subpart 
            den Median 
            \begin{solution}
                Es gilt: 
                \[ 
                    \tilde{x} = x_{\nicefrac{1}{2}} = 4
                \]
                \qed
            \end{solution}
            
            \newpage
            \subpart 
            das 10\%-Quantil
            \begin{solution}
                Es gilt: 
                \[ 
                    x_{\nicefrac{1}{10}} = 2
                \]
                \qed
            \end{solution}
            
            \subpart 
            obere Quartil
            \begin{solution}
                Es gilt: 
                \[ 
                    x_{\nicefrac{3}{4}} = 6
                \]
                \qed
            \end{solution}
        \end{subparts}
        
        \part
        Berechnen Sie die empirische Varianz und die empirische Standardabweichung.
        \begin{solution}
            Es gilt: 
            \begin{alignat*}{1}
                s^2 & = \frac{1}{n-1} \cdot \sum_{i=1}^{n} (x_i - \conj{x})^2                                             \\ 
                    & = \frac{1}{n-1} \cdot \left( \sum_{i=1}^{n} x_i^2 -n \cdot \conj{x}^2 \right)                       \\ 
                    & = \frac{1}{29} \cdot \left( \sum_{i=1}^{30} x_i^2 - 30 \cdot \left( \frac{67}{15} \right)^2 \right) \\ 
                    & = \frac{1}{29} \cdot \left( \sum_{i=1}^{30} x_i^2 - \frac{8978}{15} \right)                         \\ 
                    & = \frac{1}{29} \cdot \sum_{i=1}^{30} x_i^2 - \frac{8978}{435}                                       \\ 
                    & = \frac{1}{29} \cdot 740 - \frac{8978}{435}                                                         \\ 
                    & = \frac{2122}{435} \approx 4.878
            \end{alignat*}
            Und damit auch: 
            \[ 
                s = \sqrt{s^2} = \sqrt{\frac{2122}{435}} \approx 2.209
            \]
            \qed
        \end{solution}
    \end{parts}
\end{questions}
\end{document}