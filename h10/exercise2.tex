
\documentclass{abgabe}
\begin{document}

\begin{questions}
    \question
    Die nachfolgende Tabelle gibt eine Übersicht über die Anzahl der verkauften Bücher zu unterschiedlichen Preisen in einer Buchhandlung im Laufe eines Tages:
    \begin{center}
        \begin{tabular}{|C|C|}
            \hline
            \text{Buchpreis (in €)} & \text{Anzahl der verkauften Bücher} \\
            \hline
            \left[ 0 ; 10 \right)   & 5                                   \\
            \left[ 10 ; 30 \right)  & 15                                  \\
            \left[ 30 ; 50 \right)  & 20                                  \\
            \left[ 50 ; 80 \right)  & 12                                  \\
            \left[ 80 ; 120 \right) & 8                                   \\
            \hline
        \end{tabular}
    \end{center}

    \begin{parts}
        \part
        Berechnen Sie die jeweiligen absoluten und relativen Klassenhäufigkeiten
        \begin{solution}
            Es gilt:
            \begin{center}
                \begin{tabular}{|C|C|C|C|C|}
                    \hline
                    I_i                     & n_i & h_i              & H_i               & \nicefrac{h_i}{\abs{I_i}}        \\
                    \hline
                    \left[ 0 ; 10 \right)   & 1   & \nicefrac{1}{12} & \nicefrac{1}{12}  & \nicefrac{1}{120} \approx 0.0083 \\
                    \left[ 10 ; 30 \right)  & 4   & \nicefrac{1}{4}  & \nicefrac{1}{3}   & \nicefrac{1}{80}  = 0.0125       \\
                    \left[ 30 ; 50 \right)  & 7   & \nicefrac{1}{3}  & \nicefrac{2}{3}   & \nicefrac{1}{60}  \approx 0.0167 \\
                    \left[ 50 ; 80 \right)  & 5   & \nicefrac{1}{5}  & \nicefrac{13}{15} & \nicefrac{1}{150} \approx 0.0067 \\
                    \left[ 80 ; 120 \right) & 5   & \nicefrac{2}{15} & 1                 & \nicefrac{1}{300} \approx 0.0033 \\
                    \hline
                \end{tabular}
            \end{center}
            \qed
        \end{solution}

        \part
        Zeichnen Sie das zugehörige Histogramm.
        \begin{solution}
            \begin{center}
                % Siehe https://tex.stackexchange.com/questions/152243/bar-chart-from-csv-file-with-adjustable-bar-width?rq=1
                \begin{tikzpicture}
                    \begin{axis}
                        [
                            grid=both,
                            xlabel=$x$,
                            ylabel=$\nicefrac{h_i}{\abs{I_i}}$,
                            ybar interval,%this is the type of barplot
                            xtick=data,
                            xticklabel interval boundaries,%set the x label to be the boundaries
                            x tick label style=
                                {rotate=90,anchor=east}
                        ]
                        \addplot table [x=width, y=height, col sep=comma] {exercise2.dat}; %extract witdth and assing it to x and height and assing it to y the separator is comma take the result.csv in the same folder of your .tex when you will not use filecontents.
                    \end{axis}
                \end{tikzpicture}
            \end{center}
            \qed
        \end{solution}

        \newpage
        \part
        Bestimmen Sie
        \begin{subparts}
            \subpart
            das arithmetische Mittel,
            \begin{solution}
                Es gilt:
                \[
                    \conj{x} \approx \frac{1}{n} \cdot \sum_{i=1}^{k} n_i \cdot \alpha_i = \sum_{i=1}^{k} h_i \cdot \alpha_i = \frac{5}{12} + 5 + \frac{40}{3} + 13 + \frac{40}{3} = \frac{541}{12} \approx 45.083
                \]
                \qed
            \end{solution}

            \subpart
            den Median sowie
            \begin{solution}
                Offensichtlich ist die Einfallsklasse gegeben mit
                \[
                    I_3 = \left[ 30 ; 50 \right) = \left[ a_3 ; b_3 \right)
                \]

                Es gilt damit:
                \[
                    \tilde{x} = a_3 + \frac{\nicefrac{1}{2} - H_2}{h_3} \cdot (b_3 - a_3) = 30 + \frac{\nicefrac{1}{2} - \nicefrac{1}{3}}{\nicefrac{1}{3}}  \cdot (50 - 30) = 40
                \]
                \qed
            \end{solution}

            \subpart
            das obere und untere Quartil.
            \begin{solution}
                Offensichtlich ist die Einfallsklasse für das untere Quantil gegeben mit
                \[
                    I_2 = \left[ 10 ; 30 \right) = \left[ a_2 ; b_2 \right)
                \]
                und die für das obere Quartil mit
                \[
                    I_4 = \left[ 50 ; 80 \right) = \left[ a_4 ; b_4 \right)
                \]

                Es gilt damit:
                \[
                    x_{\nicefrac{1}{4}} = a_2 + \frac{\nicefrac{1}{4} - H_1}{h_2} \cdot (b_2 - a_2) = 10 + \frac{\nicefrac{1}{4} - \nicefrac{1}{12}}{\nicefrac{1}{4}} \cdot (30 - 10) = \frac{70}{3}
                \]
                und
                \[
                    x_{\nicefrac{3}{4}} = a_4 + \frac{\nicefrac{3}{4} - H_3}{h_4} \cdot (b_4 - a_4) = 50 + \frac{\nicefrac{3}{4} - \nicefrac{2}{3}}{\nicefrac{1}{5}} \cdot (80 - 50) = \frac{130}{2} = 62.5
                \]
                \qed
            \end{solution}
        \end{subparts}

    \end{parts}
\end{questions}
\end{document}