\documentclass{abgabe}
\begin{document}

\begin{questions}
    \question
    Die von einer Maschine für einen bestimmten Arbeitsvorgang benötigte Zeit sei eine Zufallsvariable $X$, für deren Dichtefunktion in Abhängigkeit von einem $\theta \in [0, 2]$ die Gestalt
    \[ 
        f(x;\theta) = 
        \begin{cases}
            \theta + 2(1-\theta)\cdot x & \text{für} \ x \in [0,1] \\ 
            0                           & \text{sonst}
        \end{cases}
    \]
    unterstellt wird.
    Zu $X$ liege eine einfache Stichprobe $X_1, \ldots , X_n$ (die $X_i$ sind unabhängig) vor.
    \begin{parts}
        \part
        Zeigen Sie, dass die Schätzfunktionen
        \begin{subparts}
            \subpart $\hat{\Theta}_1 = 4 - \frac{6}{n} \sum_{i=1}^{n} X_i$
            \begin{solution}
                Wir wissen, dass $\hat{\Theta}_1$ genau dann erwartungstreu ist, wenn $\Mean(\hat{\Theta}_1) = \theta$ gilt.
                
                Offensichtlich ist: 
                \begin{alignat*}{1}
                    \Mean\left( \hat{\Theta}_1 \right) & = \Mean\left( 4 - \frac{6}{n} \sum_{i=1}^{n} X_i \right)                                                              \\ 
                                                       & = 4 - \frac{6}{n} \cdot \Mean\left( \sum_{i=1}^{n} X_i \right)                                                        \\ 
                                                       & = 4 - \frac{6}{n} \cdot \sum_{i=1}^{n}  \Mean (X_i)                                                                   \\ 
                                                       & = 4 - 6 \Mean\left( X \right)                                                                                         \\ 
                                                       & = 4 - 6 \int_{-\infty}^{\infty} xf(x) \diff x                                                                         \\ 
                                                       & = 4 - 6 \left(  \int_{0}^{1} x\cdot \left( \theta + 2(1-\theta)\cdot x \right) \diff x \right)                        \\ 
                                                       & = 4 - 6 \left( \theta \int_{0}^{1} x \diff x + 2 \int_{0}^{1} x^2 \diff x - 2\theta  \int_{0}^{1} x^2 \diff x \right) \\ 
                                                       & = 4 - 6 \cdot \frac{4 - \theta}{6}                                                                                    \\ 
                                                       & = \theta
                \end{alignat*}
                \qed
            \end{solution}
            
            \newpage
            \subpart $\hat{\Theta}_2 = 3 - \frac{6}{n} \sum_{i=1}^{n} X_i^2$
            \begin{solution}
                Wir wissen, dass $\hat{\Theta}_2$ genau dann erwartungstreu ist, wenn $\Mean(\hat{\Theta}_2) = \theta$ gilt.
                
                Offensichtlich ist: 
                \begin{alignat*}{1}
                    \Mean\left( \hat{\Theta}_2 \right) & = \Mean\left( 3 - \frac{6}{n} \sum_{i=1}^{n} X_i^2   \right)                \\ 
                                                       & = 3 - \frac{6}{n} \Mean\left( \sum_{i=1}^{n} X_i^2   \right)                \\
                                                       & = 3 - 6 \Mean ( X^2 )                                                       \\
                                                       & = 3 - 6 \int_{-\infty}^{\infty} x^2f(x) \diff x                             \\
                                                       & = 3 - 6 \int_{0}^{1} x^2 \left( \theta + 2(1-\theta)\cdot x \right) \diff x \\
                                                       & = \ldots                                                                    \\
                                                       & = 3 - 6 \cdot \frac{3- \theta}{6}                                           \\
                                                       & = \theta
                \end{alignat*}
                \qed
            \end{solution}
        \end{subparts}
        erwartungstreu für $\theta$ sind.
        
        \part 
        Überprüfen Sie zusätzlich, ob $\hat{\Theta}_1$ konsistent für $\theta$ ist. 
        \begin{solution}
            Wir wissen, dass $\hat{\Theta}_1$ genau dann konsistent ist, wenn $\lim_{n\to\infty}\Var(\hat{\Theta}_1) = 0$ gilt.
            
            Offensichtlich gilt: 
            \begin{alignat*}{1}
                \lim_{n\to\infty}\Var(\hat{\Theta}_1) & = \lim_{n\to\infty}\Var(\hat{\Theta}_1)                                                    \\ 
                                                      & = \lim_{n\to\infty}\Var\left(4 - \frac{6}{n} \sum_{i=1}^{n} X_i\right)                     \\ 
                                                      & = \lim_{n\to\infty}\left( \frac{6}{n} \right)^2 \cdot \Var\left( \sum_{i=1}^{n} X_i\right) \\ 
                                                      & = 36 \cdot \lim_{n\to\infty}\frac{1}{n^2} \cdot \sum_{i=1}^{n} \Var (X_i)                  \\ 
                                                      & = 36 \cdot \lim_{n\to\infty}\frac{1}{n^2} \cdot  n \cdot  \Var (X)                         \\ 
                                                      & = 36 \cdot \lim_{n\to\infty}\frac{1}{n} \cdot  \Var (X)                                    \\ 
                                                      & = 36 \cdot 0                                                                               \\ 
                                                      & = 0
            \end{alignat*}
            
        \end{solution}
    \end{parts}
\end{questions}
\end{document}