
\documentclass{abgabe}
\begin{document}

\begin{questions}
    \question
    Peter nimmt an zehn aufeinanderfolgenden Tagen an einem Glückspiel mit einer Gewinnwahrscheinlichkeit $p \in (0, 1)$ teil.
    Dabei spielt er das Spiel jeden Tag so oft, bis er einmal gewinnt.
    Danach hört er für diesen Tag auf.
    Nach dieser Strategie verfährt er an jedem der zehn Tage.
    Nachfolgend sind die Anzahlen der Spiele angegeben, die Peter an den einzelnen Tagen spielt:

    \[
        13, 7, 10, 2, 11, 17, 15, 9, 19, 11
    \]
    \begin{parts}
        \part
        Bestimmen Sie aufgrund einer Stichprobe vom Umfang $n$ eine Maximum-Likelihood-Schätzung für die (unbekannte) Wahrscheinlichkeit $p$.
        \begin{solution}
            Wir erkennen, dass die Wahrscheinlichkeitsfunktion $f(x)$ gegeben ist mit \footnotemark
            \[
                f(x) =
                \begin{cases}
                    p(1-p)^{x-1} & \text{für} \ x \in \N \setminus {0} \\
                    0            & \text{sonst}
                \end{cases}
            \]

            Damit gilt:
            \[
                L(p) = \prod_{i=1}^{n} f(x_i) = \prod_{i=1}^{n} p(1-p)^{x_i-1} = \left( \frac{p}{1-p} \right)^n \cdot (1-p)^{\sum_{i=1}^{n} x_i}
            \]

            Damit erhalten wir $L^*(p)$ mit:
            \[
                L^*(p) = \ln L(p) = n \cdot \ln \frac{p}{1-p} + \ln (1-p) \cdot \sum_{i=1}^{n} x_i
            \]

            \[
                \implies \quad \frac{\partial L^*}{\partial p} = \frac{n}{p} + \frac{n}{1-p} - \frac{1}{1-p} \cdot \sum_{i=1}^{n} x_i = \frac{n}{p} + \frac{n - \sum_{i=1}^{n} x_i}{1-p}
            \]

            \[
                \implies \quad \frac{\partial L^*}{\partial p} = 0 \quad \iff \quad p = \frac{n}{\sum_{i=1}^{n} x_i}
            \]


            Natürlich müssen wir überprüfen, ob es sich um ein Maximum handelt.
            Es gilt offensichtlich:
            \[
                \frac{\partial^2 L^*(p)}{\partial p^2} = \frac{-n}{p^2} + \frac{n - \sum_{i=1}^{n} x_i}{(1-p)^2} = \frac{\left( \sum_{i=1}^{n} x_i \right)^3}{n^2 - n\sum_{i=1}^{n} x_i} <^{\footnotemark} 0 \quad \checkmark
            \]

            Damit haben wir einen Schätzer und es gilt:
            \[
                p = \frac{n}{\sum_{i=1}^{n} x_i}
            \]
            \qed
        \end{solution}
        \footnotetext[1]{Es handelt sich um eine verschobene geometrische Verteilung.}
        \footnotetext[2]{
            Die Summe über alle $x_i$ ist offensichtlich stets positiv.
            O.B.d.A. können wir davon ausgehen, dass in jeder ausreichend großen Stichprobe mindestens einmal mehr als einmal gespielt werden muss um zu gewinnen.
        }

        \newpage
        \part
        Berechnen Sie den zugehörigen Maximum-Likelihood-Schätzwert, der sich für die oben angegebenen Anzahlen ergibt.
        \begin{solution}
            Es gilt:
            \[
                p = \frac{n}{\sum_{i=1}^{n} x_i} = \frac{10}{114} = \frac{5}{57}
            \]
            \qed
        \end{solution}
    \end{parts}

\end{questions}
\end{document}