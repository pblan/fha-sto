
\documentclass{abgabe}
\begin{document}

\begin{questions}
    \question
    Das Umweltreferat einer Großstadt will Aufschluss darüber gewinnen, wie viele Asbestfasern pro Kubikmeter Luft im Freien in ca. einem Meter Abstand von asbestzementhaltigen Gebäudeteilen zu erwarten sind.
    Bei $n = 14$ diesbezüglichen Messungen traten die Werte
    \begin{center}
        \begin{tabular}{ccccccc}
            980  & 1340 & 610  & 750 & 880  & 1250 & 2410 \\
            1100 & 470  & 1040 & 910 & 1860 & 730  & 820
        \end{tabular}
    \end{center}
    auf, die als Ergebnisse unabhängiger normalverteilter Stichprobenvariablen angesehen werden.
    \begin{parts}
        \part
        Führen Sie für den Erwartungswert $\mu$ der Anzahl $X$ der unter den obigen Bedingungen vorhandenen Asbestfasern eine Intervallschätzung zum Konfidenzniveau $0.95$ durch.
        \begin{solution}
            Wir haben offensichtlich das Vertrauensniveau gegeben mit
            \[
                \gamma = 1 - \alpha = 0.95 \implies \alpha = 0.05
            \]

            Wir wissen, dass gilt:
            \[
                X \sim \mathcal{N}(\mu, \sigma^2)
            \]

            Ein geeigneter Schätzer für $\mu$ ist bekanntermaßen
            \[
                \conj{x} = \frac{1}{n} \cdot \sum_{i=1}^{n} x_i
            \]

            Ein geeigneter Schätzer für $\sigma$ ist bekanntermaßen
            \[
                s = \sqrt{\frac{1}{n-1} \cdot \sum_{i=1}^{n} (x_i - \conj{x})^2}
            \]



            Für unsere Verteilung gilt dann:
            \[
                \conj{x} = \frac{1}{14} \cdot 15150 = \frac{7575}{7}
            \]
            und
            \[
                s = \sqrt{\frac{1}{13} \cdot \frac{24126450}{7}} = \frac{5\sqrt{87820278}}{91}
            \]

            Die Zufallsvariable $T$ mit
            \[
                T = \sqrt{n} \cdot \frac{\conj{x} - \mu}{s}
            \]
            ist dann $t$-verteilt mit $n-1$ Freiheitsgraden.

            Es muss gelten:
            \[
                P(-c \leq T \leq c) = \gamma = 1 - \alpha
            \]

            Dann gilt:
            \begin{alignat*}{3}
                             & -c                                                                             & \leq T                                       & \leq c                                                                              \\
                \equiv \quad & -t_{n-1} \left( 1-\frac{\alpha}{2} \right)                                     & \leq T                                       & \leq t_{n-1} \left( 1-\frac{\alpha}{2} \right)                                      \\
                \equiv \quad & -t_{n-1} \left( 1-\frac{\alpha}{2} \right)                                     & \leq \sqrt{n} \cdot \frac{\conj{x} - \mu}{s} & \leq t_{n-1} \left( 1-\frac{\alpha}{2} \right)                                      \\
                \equiv \quad & -t_{n-1} \left( 1-\frac{\alpha}{2} \right) \cdot \frac{s}{\sqrt{n}} - \conj{x} & \leq  - \mu                                  & \leq t_{n-1} \left( 1-\frac{\alpha}{2} \right) \cdot \frac{s}{\sqrt{n}}  - \conj{x} \\
                \equiv \quad & \conj{x} + t_{n-1} \left( 1-\frac{\alpha}{2} \right) \cdot \frac{s}{\sqrt{n}}  & \geq  \mu                                    & \geq \conj{x} - t_{n-1} \left( 1-\frac{\alpha}{2} \right) \cdot \frac{s}{\sqrt{n}}  \\
                \equiv \quad & \conj{x} - t_{n-1} \left( 1-\frac{\alpha}{2} \right) \cdot \frac{s}{\sqrt{n}}  & \leq  \mu                                    & \leq \conj{x} + t_{n-1} \left( 1-\frac{\alpha}{2} \right) \cdot \frac{s}{\sqrt{n}}
            \end{alignat*}

            Damit gilt:
            \begin{alignat*}{2}
                             & P(\conj{x} - t_{n-1} \left( 1-\frac{\alpha}{2} \right) \cdot \frac{s}{\sqrt{n}} \leq  \mu \leq \conj{x} + t_{n-1} \left( -\frac{\alpha}{2} \right) \cdot \frac{s}{\sqrt{n}})                                    & = \gamma = 1 - \alpha \\
                \equiv \quad & P(\frac{7575}{7} - t_{13} \left( 0.975 \right) \cdot \frac{5\sqrt{87820278}}{91 \cdot \sqrt{14}} \leq  \mu \leq \frac{7575}{7} + t_{13} \left( 0.975 \right) \cdot \frac{5\sqrt{87820278}}{91 \cdot \sqrt{14}}) & = 0.95                \\
                \equiv \quad & P(\frac{7575}{7} - 2.160 \cdot \frac{5\sqrt{87820278}}{91 \cdot \sqrt{14}} \leq  \mu \leq \frac{7575}{7} + 2.160 \cdot \frac{5\sqrt{87820278}}{91 \cdot \sqrt{14}})                                             & = 0.95                \\
                \equiv \quad & P(784.897 \leq  \mu \leq 1379.389)                                                                                                                                                                              & = 0.95                \\
            \end{alignat*}

            Damit erhalten wir unser Konfidenzintervall $I_{14}$ mit
            \[
                I_{14} = [784.897, 1379.389]
            \]
            \qed
        \end{solution}

        \part
        Wie müsste das Konfidenzniveau gewählt sein, damit die Länge des entstehenden Schätzintervalls gleich 500 ist?
        \begin{solution}
            Offensichtlich gilt für die Intervallbreite $I_B$:
            \[
                I_B := c_o - c_u = \conj{x} + t_{n-1} \left( 1-\frac{\alpha}{2} \right) \cdot \frac{s}{\sqrt{n}} - \left( \conj{x} - t_{n-1} \left( 1-\frac{\alpha}{2} \right) \cdot \frac{s}{\sqrt{n}} \right) = 2 t_{n-1} \left( 1-\frac{\alpha}{2} \right) \cdot \frac{s}{\sqrt{n}}
            \]

            Es ergibt sich dann:
            \begin{alignat*}{2}
                               & I_B                                                                  &  & = 500                                                   \\
                \equiv \quad   & 2 t_{n-1} \left( 1-\frac{\alpha}{2} \right) \cdot \frac{s}{\sqrt{n}} &  & = 500                                                   \\
                \equiv \quad   & t_{n-1} \left( 1-\frac{\alpha}{2} \right) \cdot \frac{s}{\sqrt{n}}   &  & = 250                                                   \\
                \equiv \quad   & t_{n-1} \left( 1-\frac{\alpha}{2} \right)                            &  & = 250 \cdot \frac{\sqrt{n}}{s}                          \\
                \equiv \quad   & t_{13} \left( 1-\frac{\alpha}{2} \right)                             &  & = 250 \cdot \frac{\sqrt{14} \cdot 91}{5\sqrt{87820278}} \\
                \implies \quad & t_{13} \left( 1-\frac{\alpha}{2} \right)                             &  & \approx 1.8167                                          \\
            \end{alignat*}

            Der nächste Wert in der gegebenen Tabelle der Vorlesung für $t_{13} \left( 1-\frac{\alpha}{2} \right)$ wäre
            \[
                t_{13}(0.95) = 1.771
            \]

            Stellen wir nun um, erhalten wir:
            \[
                1 - \frac{\alpha}{2} = 0.95 \implies \alpha = 0.1
            \]

            Damit erhalten wir ungefähr ein Konfidenzniveau von
            \[
                \gamma = 1 -\alpha = 1 - 0.1 = 0.9
            \]
            \qed
        \end{solution}
    \end{parts}
\end{questions}
\end{document}