
\documentclass{abgabe}
\begin{document}

\begin{questions}
    \question
    Wie viele verschiedene Moeglichkeiten gibt es, 7 Studierende auf 7 Plaetzen anzuordnen, wenn diese eine Sitzreihe im Hoersaal bilden?
    Was aendert sich, wenn die Platzierung auf 7 Stuehlen um einen runden Tisch geschehen soll?
    \begin{solution}
        Fuer die Sitzreihe gilt trivialerweise: (Permutation)
        \[
            P(7) = 7! = \num{5040}  
        \]

        Fuer den runden Tisch gilt dann, da die Wahl der \glqq ersten Position\grqq egal ist: (Permutation)
        \[
            \frac{P(7)}{7} = P(6) = 720
        \]\qed
    \end{solution}
\end{questions}
\end{document}