
\documentclass{abgabe}
\begin{document}

\begin{questions}
    \question
    Wie viele verschiedene Möglichkeiten gibt es, 7 Studierende auf 7 Plätzen anzuordnen, wenn diese eine Sitzreihe im Hörsaal bilden?
    Was ändert sich, wenn die Platzierung auf 7 Stühlen um einen runden Tisch geschehen soll?
    \begin{solution}
        Für die Sitzreihe gilt trivialerweise: (Permutation)
        \[
            P(7) = 7! = \num{5040}  
        \]

        Für den runden Tisch gilt dann, da die Wahl der \glqq ersten Position\grqq egal ist: (Permutation)
        \[
            \frac{P(7)}{7} = P(6) = 720
        \]\qed
    \end{solution}
\end{questions}
\end{document}