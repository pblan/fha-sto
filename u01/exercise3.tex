
\documentclass{abgabe}
\begin{document}

\begin{questions}
    \question
    Zu einer Feier wollen die Gaeste Weisswein trinken. 
    Von 3 Sorten stehen jeweils 12 nicht unterscheidbare Flaschen im Keller. 
    \begin{parts}
        \part 
        Der Kuehlschrank im Keller fasst 6 Flaschen.
        Wie viele Moeglichkeiten gibt es aus den 3 Sorten den Kuehlschrank zu bestuecken?
        \begin{solution}
            Es gilt: (Kombination mit Wiederholung)
            \[
                C_w(3;6) = \binom{3 + 6 - 1}{6} = \binom{8}{6} = \frac{8!}{6! \cdot 2!} = 28 
            \]
            \qed
        \end{solution}

        \part 
        Wie viele Anordnungen der Flaschen im Regal gibt es, wenn die Flaschen einer Sorte nicht unterscheidbar sind? 
        \begin{solution}
            Es gilt: (Permutation mit nicht unterscheidbaren Flaschen) 
            \[ 
                P(36;12;12;12) = \frac{36!}{12! \cdot 12! \cdot 12!} \approx 3.38 \cdot 10^{15}
            \]
            \qed
        \end{solution}
    \end{parts}
\end{questions}
\end{document}