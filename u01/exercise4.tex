
\documentclass{abgabe}
\begin{document}

\begin{questions}
    \question
    Eine Lieferung von 30 Geräten, die durch ihre Fabrikationsnummern unterscheidbar sind, enthält 6 fehlerhafte Geräte.
    \begin{parts}
        \part 
        Wie viele verschiedene Stichproben des Umfangs 5 sind möglich?
        \begin{solution}
            Es gilt: (Kombination ohne Wiederholung)
            \[ 
                C(30;5) = \binom{30}{5} = \frac{30!}{5! \cdot 25!} = \num{142506}
            \]
            \qed
        \end{solution}

        \part 
        Wie viele Stichproben des Umfangs 5 mit genau 2 fehlerhaften Geräten sind möglich?
        \begin{solution}
            Es gilt: (Kombination ohne Wiederholung pro Gerätetyp)
            \[ 
                C(24;3) \cdot C(6;2) = \binom{24}{3} \cdot \binom{6}{2} = \ldots = \num{30360}
            \] 
            \qed
        \end{solution}

        \part 
        Wie viele Stichproben des Umfangs 5 mit höchstens einem fehlerhaften Gerät sind möglich?
        \begin{solution}
            1. Situation: 
            \begin{itemize}
                \item Alle 5 Geräte kommen aus den 24 funktionstüchtigen Geräten.
                \item Keine Geräte kommen aus den 6 defekten Geräten.
            \end{itemize}

            2. Situation: 
            \begin{itemize}
                \item 4 Geräte kommen aus den 24 funktionstüchtigen Geräten.
                \item 1 Gerät kommt aus den 6 defekten Geräten.
            \end{itemize}

            Es gilt: (Kombination ohne Wiederholung pro Gerätetyp)
            \[ 
                n_1 = C(24;5) \cdot C(6;0) = \binom{24}{5} \cdot \binom{6}{0} = \ldots = \num{42504}
            \] 
            \[ 
                n_2 = C(24;4) \cdot C(6;1) = \binom{24}{4} \cdot \binom{6}{1} = \ldots = \num{63756}
            \] 

            Insgesamt gilt damit:
            \[ 
                n = n_1 + n_2 = \num{106260}
            \]
            \qed
        \end{solution}
    \end{parts}

    Dabei wird, wie in der Praxis üblich, eine geprüfte Einheit nach der Prüfung nicht in das Lieferlos zurückgelegt.
\end{questions}
\end{document}