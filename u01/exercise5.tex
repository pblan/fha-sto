
\documentclass{abgabe}
\begin{document}

\begin{questions}
    \question
    Ein Zweig des Telefonnetzes wird ueber sechsstellige Nummern angewaehlt. 
    Die erste Ziffer der Telefonnummer darf keine Null sein. 
    Fuer wieviele Anschluesse reichen die moeglichen Kombinationen aus?
    \begin{solution}
        Es gilt: (Variation mit Wiederholung fuer 1. Ziffer bzw. 2.-6. Ziffer)
        \[ 
            V_w(9;1) \cdot V_w(10;5) = 9^1 \cdot 10^5 = \num{900000}
        \] 
        \qed
    \end{solution}
\end{questions}
\end{document}