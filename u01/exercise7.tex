
\documentclass{abgabe}
\begin{document}

\begin{questions}
    \question
    Berechnen Sie, wie lange ein Skatspieler leben muesste, wenn er 
    \begin{parts}
        \part 
        alle moeglichen Blaetter aus 10 Karten auf die Hand (ein Skatspielbesteht aus 32 Karten) bekommt.
        \begin{solution}
            Es gilt: (Kombination ohne Wiederholung)
            \[ 
                n = C(32;10) = \binom{32}{10} = \frac{32!}{10! \cdot 22!} = \num{64512240}
            \] 
            Da der Skatspieler 5min pro Spiel braucht, benoetigt er: 
            \[ 
                n \cdot 5\si{\min} = \num{64512240} \cdot 5\si{\min} = \num{322561200}\si{\min} \approx \num{224000}\si{\day}
            \] 
            \qed
        \end{solution}

        \part 
        alle moeglichen Spiele, also auch alle moeglichen Blaetter seiner zwei Mitspieler und der zwei Skatkarten beruecksichtigt werden. 
        \begin{solution}
            Es gilt: (Kombinationen ohne Wiederholung)
            \[ 
                \begin{aligned}
                    n ={} & C(32;10) \cdot C(22;10) \cdot C(12;10) \cdot C(2;2) \\ 
                    ={} & \binom{32}{10} \cdot \binom{22}{10} \cdot \binom{12}{10} \cdot \binom{2}{2} \\ 
                    ={} & \ldots \\ 
                    ={} & \num{2753294408504640}
                \end{aligned}
            \] 

            Analog wuerde er hierfuer ca. \num{9560050029530} Tage benoetigen.
            \qed
        \end{solution}
    \end{parts}

    \emph{Annahme:} Ein Spiel dauert ca. 5 Minuten und er macht nichts anderes als spielen, spielen \ldots
\end{questions}
\end{document}