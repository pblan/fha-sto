
\documentclass{abgabe}
\begin{document}

\begin{questions}
    \question
    Fuer die Klaerung einer Sachfrage will ein Ausschuss aus seinen 14 Mitgliedern einen \gqq{Fuenferrat} bilden. 
    Dieser ist jedoch arbeitsunfaehig, wenn ihm 2 bestimmte Ausschussmitglieder gleichzeitig angehoeren, da diese beiden sich nicht moegen.
    Wie viele Moeglichkeiten gibt es fuer die Bildung des \gqq{Fuenferrats}?
    \begin{solution}
        Wir koennen uns ohne Probleme Fuenferteams aus den 12 unproblematischen Mitgliedern bilden. 
        Um dann die beiden Streithaehne auch unterzubringen, berechnen wir die Anzahl an Viererteams aus den selben 12 Mitgliedern, verdoppeln diese Zahl und haben so die gesamte Anzahl an Moeglichkeiten.

        Es gilt: (Kombinationen ohne Wiederholung)
        \[ 
            n = C(12;5) + 2 \cdot C(12;4) = \binom{12}{5} + 2 \cdot \binom{12}{4} = \ldots = \num{1782}
        \] 
        \qed
    \end{solution}
\end{questions}
\end{document}