
\documentclass{abgabe}
\begin{document}

\begin{questions}
    \question
    Fuer die Klärung einer Sachfrage will ein Ausschuss aus seinen 14 Mitgliedern einen \gqq{Fünferrat} bilden. 
    Dieser ist jedoch arbeitsunfähig, wenn ihm 2 bestimmte Ausschussmitglieder gleichzeitig angehören, da diese beiden sich nicht mögen.
    Wie viele Möglichkeiten gibt es für die Bildung des \gqq{Fünferrats}?
    \begin{solution}
        Wir können uns ohne Probleme Fünferteams aus den 12 unproblematischen Mitgliedern bilden. 
        Um dann die beiden Streithähne auch unterzubringen, berechnen wir die Anzahl an Viererteams aus den selben 12 Mitgliedern, verdoppeln diese Zahl und haben so die gesamte Anzahl an Möglichkeiten.

        Es gilt: (Kombinationen ohne Wiederholung)
        \[ 
            n = C(12;5) + 2 \cdot C(12;4) = \binom{12}{5} + 2 \cdot \binom{12}{4} = \ldots = \num{1782}
        \] 
        \qed
    \end{solution}
\end{questions}
\end{document}