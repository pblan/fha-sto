
\documentclass{abgabe}
\begin{document}

\begin{questions}
    \question
    Eine technische Anlage besteht aus drei Baugruppen, die zufällig und unabhängig voneinander arbeitsfähig oder defekt sein können. 
    Wir registrieren die Zustände der drei Baugruppen und wählen als einfache Darstellung eine \gqq{1} für eine arbeitsfähige Baugruppe und eine \gqq{0} für eine defekte. 
    Geben Sie für die folgenden Aufbauten die Teilmenge von $\Omega$ an, die das Ergebnis $A = \{ \text{Die Anlage ist arbeitsfähig} \}$ darstellt. 
    
    Formulieren Sie die Ereignisse zunächst als Mengenverknüpfung;
    
    $B_i = \{ i\text{-te Baugruppe ist arbeitsfähig} \}, i = 1,2,3$
    \begin{parts}
        \part 
        Die Anlage sei genau dann arbeitsfähig, wenn alle drei Baugruppen arbeitsfähig sind. 
        
        (In der Zuverlässigkeitstheorie stellt man einen solchen Aufbau schematisch als Reihenschaltung der drei Baugruppen dar.)
        \begin{solution}
            Es gilt:
            \[ 
                A = B_1 \cap B_2 \cap B_3
            \]
            \qed
        \end{solution}
        
        \part 
        Die Anlage genau dann arbeitsfähig ist, wenn die 1. und 2. Baugruppe oder die 1. und 3. Baugruppe oder alle drei Baugruppen arbeitsfähig sind. 
        
        (In der Zuverlässigkeitstheorie entspricht dieser Aufbau einer Parallelschaltung von von $B_2$ und $B_3$, zu der $B_1$ in Reihe gelegt wurde.)
        \begin{solution}
            Es gilt:
            \[ 
                A = (B_1 \cap B_2) \cup (B_1 \cap B_3) = B_1 \cap (B_2 \cup B_3)
            \]
            
            \qed
        \end{solution}
        
        \part 
        Die Anlage genau dann arbeitsfähig ist, wenn mindestens ein Bauteil arbeitsfähig ist. 
        
        (In der Zuverlässigkeitstheorie entspricht dieser Aufbau einer Parallelschaltung von $B_1$, $B_2$ und $B_3$.)
        \begin{solution}
            Es gilt:
            \[ 
                A = B_1 \cup B_2 \cup B_3
            \]
            
            \qed
        \end{solution}
    \end{parts}
\end{questions}
\end{document}