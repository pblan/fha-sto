
\documentclass{abgabe}
\begin{document}

\begin{questions}
    \question
    Bei der Produktion von Federn treten folgende Fehler auf ($\si{ppm} \ \widehat{=} \ 10^{-6}$)
    
    \[ 
        \begin{aligned}
             & F = \{ \text{Federkonstante zu klein} \}              &  & \qquad \text{Anteil} \  0.8\%      \\
             & D = \{ \text{Durchmesser falsch} \}                   &  & \qquad \text{Anteil} \  0.5\%      \\
             & \text{Federkonstante zu klein und Durchmesser falsch} &  & \qquad \text{Anteil} \  40\si{ppm} \\
        \end{aligned}
    \]
    \begin{parts}
        \part 
        Wie groß ist der Anteil fehlerhafter Federn an der Gesamtproduktion, wenn die Fehler unabhängig voneinander auftreten, sich aber gegenseitig nicht ausschließen? 
        \begin{solution}
            Es gilt: 
            \[ 
                P(F \cup D) = P(F) + P(D) - P(F \cap D) = 0.8\% + 0.5\% - 0.004\% = 1.296\%
            \]
            \qed
        \end{solution}
        
        \part 
        Wie groß ist der Anteil fehlerhafter Federn, die nur einen falschen Durchmesser haben?
        \begin{solution}
            Es gilt: 
            \[ 
                P(D \cap \conj{F}) = P(D) - P(F \cap D) = 0.5\% - 0.004\% = 0.496\% 
            \]
            \qed
        \end{solution}
    \end{parts}
\end{questions}
\end{document}