
\documentclass{abgabe}
\begin{document}

\begin{questions}
    \question
    In einer Lieferung von 10 hochwertigen Geräten befinden sich 2 defekte Geräte. 
    Als Eingangskontrolle wurde vereinbart, dass der Abnehmer 5 Geräte zufällig entnimmt und auf Funktionstüchtigkeit überprüft. 
    Befindet sich in dieser Stichprobe höchstens eine fehlerhafte Einheit, wird die Lieferung entnommen, andernfalls an den Lieferanten zur Sortierprüfung zurückgeschickt. 
    Die geprüften Einheiten werden, wie in der Praxis üblich, nach der Prüfung nicht in das Lieferlos zurückgelegt. 
    Wie wahrscheinlich ist es, dass diese Lieferung vom Abnehmer akzeptiert wird? 
    
    \emph{Tipp:} Berechnen Sie zunächst die Mächtigkeit der Ergebnismenge (Zahl möglicher Stichproben)
    \begin{solution}
        Sei $A = \{ \text{Es wird bei der Stichprobe höchstens eine fehlerhafte Einheit entnommen} \}$.
        
        Es gelten: (Kombinationen ohne Wiederholung)
        \[ 
            \abs{\Omega} = \binom{10}{5} = 252 \qquad \left( = C(8;5) + C(8;4) \cdot C(2;1) + C(8;3) \cdot C(2;2) \right)
        \]
        \[ 
            \abs{A} = C(8;5) + C(8;4) \cdot C(2;1) = \binom{8}{5} + \binom{8}{4} \cdot \binom{2}{1} = 56 + 70 \cdot 2 = 196
        \]
        
        Damit gilt dann: 
        \[ 
            P(A) = \frac{\abs{A}}{\abs{\Omega}} = \frac{196}{252} \approx 0.762
        \]
        \qed
    \end{solution}
\end{questions}
\end{document}