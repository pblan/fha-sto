
\documentclass{abgabe}
\usepackage[linguistics]{forest}
\usepackage{nicefrac}

\begin{document}

\forestset{
  left prob/.style={
      edge label={node[midway, above left] {\tiny #1}}
    },
  right prob/.style={
      edge label={node[midway, above right] {\tiny #1}}
    },
  marked/.style={
      edge=red, text=red
    }
}

\begin{questions}
  \question
  Bei der Fertigung eines Loses Elektronenröhren in der Probefertigung treten drei Fehlerarten auf:
  \[ 
    \begin{aligned}
       & F_1 = \{ \text{zu niedrige Kathodenemission} \} &  & \qquad \text{Anteil} \  15\% \\
       & F_2 = \{ \text{Schluss} \}                      &  & \qquad \text{Anteil} \  5\%  \\
       & F_3 = \{ \text{Isolationsfehler} \}             &  & \qquad \text{Anteil} \  10\% \\
    \end{aligned}
  \]
  Die Entstehung der verschiedenen Fehlerarten ist völlig unabhängig voneinander, die Fehler schließen sich aber gegenseitig nicht aus.
  \begin{parts}
    \part 
    Wie groß ist der Anteil fehlerhafter Röhren?
    \begin{solution}
      \begin{center}

        \begin{forest}
          [
            [
                $F_1$, left prob={$15\%$}, marked
                  [
                    $F_2$, left prob={$5\%$}, marked
                      [
                        $F_3$, left prob={$10\%$}, marked
                          [
                            $0.075\%$, edge={->}, marked
                          ]
                      ]
                      [
                        $\conj{F_3}$, right prob={$90\%$}, marked
                          [
                            $0.675\%$, edge={->}, marked
                          ]
                      ]
                  ]
                  [
                    $\conj{F_2}$, right prob={$95\%$}, marked
                      [
                        $F_3$, left prob={$10\%$}, marked
                          [
                            $1.425\%$, edge={->}, marked
                          ]
                      ]
                      [
                        $\conj{F_3}$, right prob={$90\%$}, marked
                          [
                            $12.825\%$, edge={->}, marked
                          ]
                      ]
                  ]
              ]
              [
                $\conj{F_1}$, right prob={$85\%$}, marked
                  [
                    $F_2$, left prob={$5\%$}, marked
                      [
                        $F_3$, left prob={$10\%$}, marked
                          [
                            $0.425\%$, edge={->}, marked
                          ]
                      ]
                      [
                        $\conj{F_3}$, right prob={$90\%$}, marked
                          [
                            $3.825\%$, edge={->}, marked
                          ]
                      ]
                  ]
                  [
                    $\conj{F_2}$, right prob={$95\%$}, marked
                      [
                        $F_3$, left prob={$10\%$}, marked
                          [
                            $8.075\%$, edge={->}, marked
                          ]
                      ]
                      [
                        $\conj{F_3}$, right prob={$90\%$}
                          [
                            $72.675\%$, edge={->}
                          ]
                      ]
                  ]
              ]
          ]
        \end{forest}
      \end{center}
      
      Wir wissen, dass für $A = \{ \text{Fehlerhafte Röhre} \}$ gilt: 
      \[ 
        P(A) = 1 - P(\conj{F_1} \cap \conj{F_2} \cap \conj{F_3}) = 1 - 72.675\% = 27.325\%
      \]
      \qed
    \end{solution}
    
    \part 
    Wie groß ist der Anteil der Röhren, die alle drei Fehlerarten aufweisen?
    \begin{solution}
      \begin{center}

        \begin{forest}
          [
            [
                $F_1$, left prob={$15\%$}, marked
                  [
                    $F_2$, left prob={$5\%$}, marked
                      [
                        $F_3$, left prob={$10\%$}, marked
                          [
                            $0.075\%$, edge={->}, marked
                          ]
                      ]
                      [
                        $\conj{F_3}$, right prob={$90\%$}
                          [
                            $0.675\%$, edge={->}
                          ]
                      ]
                  ]
                  [
                    $\conj{F_2}$, right prob={$95\%$}
                      [
                        $F_3$, left prob={$10\%$}
                          [
                            $1.425\%$, edge={->}
                          ]
                      ]
                      [
                        $\conj{F_3}$, right prob={$90\%$}
                          [
                            $12.825\%$, edge={->}
                          ]
                      ]
                  ]
              ]
              [
                $\conj{F_1}$, right prob={$85\%$}
                  [
                    $F_2$, left prob={$5\%$}
                      [
                        $F_3$, left prob={$10\%$}
                          [
                            $0.425\%$, edge={->}
                          ]
                      ]
                      [
                        $\conj{F_3}$, right prob={$90\%$}
                          [
                            $3.825\%$, edge={->}
                          ]
                      ]
                  ]
                  [
                    $\conj{F_2}$, right prob={$95\%$}
                      [
                        $F_3$, left prob={$10\%$}
                          [
                            $8.075\%$, edge={->}
                          ]
                      ]
                      [
                        $\conj{F_3}$, right prob={$90\%$}
                          [
                            $72.675\%$, edge={->}
                          ]
                      ]
                  ]
              ]
          ]
        \end{forest}
      \end{center}
      
      Es gilt: 
      \[ 
        P(F_1 \cap F_2 \cap F_3) = 0.075\% 
      \]
      \qed
    \end{solution}
  \end{parts}
\end{questions}
\end{document}