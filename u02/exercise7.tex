
\documentclass{abgabe}
\begin{document}

\begin{questions}
    \question
    Bei der Fertigung eines Loses Elektronenröhren in der Probefertigung treten drei Fehlerarten auf:
    \[ 
        \begin{aligned}
             & F_1 = \{ \text{zu niedrige Kathodenemission} \} &  & \qquad \text{Anteil} \  15\% \\
             & F_2 = \{ \text{Schluss} \}                      &  & \qquad \text{Anteil} \  5\%  \\
             & F_3 = \{ \text{Isolationsfehler} \}             &  & \qquad \text{Anteil} \  10\% \\
        \end{aligned}
    \]
    Die Entstehung der verschiedenen Fehlerarten ist völlig unabhängig voneinander, die Fehler schließen sich aber gegenseitig nicht aus.
    \begin{parts}
        \part 
        Wie groß ist der Anteil fehlerhafter Röhren?
        \begin{solution}

            \qed
        \end{solution}
        
        \part 
        Wie groß ist der Anteil der Röhren, die alle drei Fehlerarten aufweisen?
        \begin{solution}

            \qed
        \end{solution}
    \end{parts}
\end{questions}
\end{document}