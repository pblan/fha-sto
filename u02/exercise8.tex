
\documentclass{abgabe}
\begin{document}

\newcommand{\Aalt}{A_{\text{alt}}}
\newcommand{\Aneu}{A_{\text{neu}}}

\begin{questions}
    \question
    Zwei Abwasserpumpen arbeiten völlig unabhängig voneinander (Redundanz). 
    Nach Auswertung der Wartungshefte zeigt sich, dass die neue Pumpe eine Ausfallwahrscheinlichkeit von 5\%, die ältere von 10\% hat. 
    Die Wahrscheinlichkeit für den gleichzeitigen Ausfall beider Pumpen beträgt 0.5\%.
    Da ein Notbetrieb mit einer Pumpe nur kurzzeitig möglich ist, ist die Wahrscheinlichkeit für das Eintreten dieses Notbetriebes gesucht.
    \begin{solution}
        Es gilt:  
        \[ 
            P(\Aalt \cup \Aneu) = P(\Aalt) + P(\Aneu) - P(\Aalt \cap \Aneu) = 10\% + 5\% - 0.5\% = 14.5\%
        \]
        und damit:
        \[ 
            P((\Aalt \cup \Aneu) \cap \conj{(\Aalt \cap \Aneu)}) = P(\Aalt \cup \Aneu) - P(\conj{\Aalt \cap \Aneu}) = 14.5\% - 0.5\% = 14\%
        \]
        \qed
    \end{solution}
\end{questions}
\end{document}