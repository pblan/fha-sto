
\documentclass{abgabe}
\begin{document}

\begin{questions}
    \question
    Bei der Fertigung eines Produktes sei bekannt, dass 96\%der produzierten Einheiten die geforderten Spezifikationen einhalten. 
    Die bislang unvermeidbaren 4\% fehlerhaften Einheiten müssen durch eine Prüfung erkannt und vor der Auslieferung aussortiert werden.
    Da das bisher verwendete Prüfverfahren sehr teuer ist, soll eine preiswerte Alternative getestet werden. 
    Dieses billigere Verfahren erkennt allerdings nur 98\% der brauchbaren Einheiten als brauchbar und stuft 5\% der defekten Stücke als brauchbar ein. 
    Um zu einer Entscheidung über die Einführung dieses billigeren Verfahrens zu kommen, sollen folgende Werte berechnet werden:
    \begin{parts}
        \part
        Anteil der insgesamt als defekt eingestuften Einheiten
        \begin{solution}

        \end{solution}
        
        \part 
        Anteil der fehlerfreien unter den als fehlerfrei eingestuften Einheiten
        \begin{solution}

        \end{solution}
        
        \part 
        Anteil der fehlerfreien unter den als defekt eingestuften Einheiten
        \begin{solution}

        \end{solution}
    \end{parts}
\end{questions}
\end{document}