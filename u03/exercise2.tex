
\documentclass{abgabe}
\begin{document}

\begin{questions}
    \question
    Für die Umfrage \gqq{Haben Sie schon einmal einen Ladendiebstahl begangen} wurde wie folgt vorgegangen: 
    Jeder Befragte würfelte; 
    das Ergebnis des einfachen Würfelwurfes war dem Interviewer nicht bekannt. 
    Bei dem Würfelergebnis 1, 2 oder 3 antwortete der Befragte wahrheitsgemäß mit \gqq{Ja} oder \gqq{Nein}. 
    Bei den Augenzahlen 4 oder 5 antwortete er stets mit \gqq{Ja}, bei einer 6 als Würfelergebnis stets mit \gqq{Nein}. 
    Das Ergebnis der Umfrage war, dass 384 der 1033 Befragten mit \gqq{Ja} antworteten.
    \begin{parts}
        \part
        Zeichnen Sie für dieses zweistufige Zufallsexperiment einen Wahrscheinlichkeitsbaum und vergeben Sie für die gesuchte, unbekannte Wahrscheinlichkeit, dass ein Ladendiebstahl begangen wurde, einen Variablennamen (z.B. $x$).
        \begin{solution}

        \end{solution}
        
        \part 
        Berechnen Sie die totale Wahrscheinlichkeit für die Antwort \gqq{Ja} (das Ergebnis ist keine Zahl, da $x$ aus (a) noch unbekannt ist).
        \begin{solution}

        \end{solution}
        
        \part 
        Schätzen Sie die unbekannte Wahrscheinlichkeit für einen Ladendiebstahl, indem Sie das Ergebnis der Umfrage verwenden.
        \begin{solution}

        \end{solution}
    \end{parts}
\end{questions}
\end{document}