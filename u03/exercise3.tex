
\documentclass{abgabe}
\begin{document}

\begin{questions}
    \question
    Für $n = 0, 1, 2, \ldots, 49 $ sei $A_n$ das Ereignis, dass Matse-Azubi Karl genau $n$ der 49 Übungsaufgaben zur Stochastik vor der Klausur selbstständig bearbeitet hat.
    $B$ sei das Ereignis, dass Karl die Klausur besteht. 
    Zur Vereinfachung sei angenommen:
    $P(A_n) = \frac{1}{50}$ und die Wahrscheinlichkeit, mit der er besteht, wenn er $n$ Aufgaben bearbeitet hat, sei $\frac{n}{50}$.
    \begin{parts}
        \part
        Berechnen Sie die Wahrscheinlichkeit, dass Karl die Klausur besteht.
        \begin{solution}

        \end{solution}
        
        \part 
        Karl hat die Klausur bestanden. 
        Wie groß ist die Wahrscheinlichkeit, dass er vor der Klausur nicht mehr als 5 Aufgaben selbstständig bearbeitet hat?
        \begin{solution}

        \end{solution}
    \end{parts}
\end{questions}
\end{document}