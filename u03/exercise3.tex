
\documentclass{abgabe}
\begin{document}

\begin{questions}
    \question
    Für $n = 0, 1, 2, \ldots, 49 $ sei $A_n$ das Ereignis, dass Matse-Azubi Karl genau $n$ der 49 Übungsaufgaben zur Stochastik vor der Klausur selbstständig bearbeitet hat.
    $B$ sei das Ereignis, dass Karl die Klausur besteht. 
    Zur Vereinfachung sei angenommen:
    $P(A_n) = \frac{1}{50}$ und die Wahrscheinlichkeit, mit der er besteht, wenn er $n$ Aufgaben bearbeitet hat, sei $\frac{n}{50}$.
    \begin{parts}
        \part
        Berechnen Sie die Wahrscheinlichkeit, dass Karl die Klausur besteht.
        \begin{solution}
            Es gilt: 
            \begin{center}
                \begin{forest}
                    [, wide
                            [
                                $A_0$, above left prob={$\nicefrac{1}{50}$}
                                    [
                                        $B$, left prob={$\nicefrac{0}{50}$}
                                    ]
                            ]
                            [
                                $A_1$
                                [
                                        $B$, left prob={$\nicefrac{1}{50}$}
                                    ]
                            ]
                            [
                                $A_2$
                                [
                                        $B$, left prob={$\nicefrac{2}{50}$}
                                    ]
                            ]
                            [
                                $A_3$
                                [
                                        $B$, left prob={$\nicefrac{3}{50}$}
                                    ]
                            ]
                            [
                                $A_4$
                                [
                                        $B$, left prob={$\nicefrac{4}{50}$}
                                    ]
                            ]
                            [
                                $A_5$
                                [
                                        $B$, left prob={$\nicefrac{5}{50}$}
                                    ]
                            ]
                            [
                                $\ldots$
                                [
                                        $\ldots$
                                    ]
                            ]
                            [
                                $A_{49}$, above right prob={$\nicefrac{1}{50}$}
                                    [
                                        $B$, left prob={$\nicefrac{49}{50}$}
                                    ]
                            ]
                    ]
                \end{forest}
            \end{center}
            
            Und damit
            \[ 
                P(B) = \sum_{i = 0}^{49} P(A_i \cap B) = \frac{1}{50^2} \cdot \sum_{i = 0}^{49} i = \frac{1}{2500} \cdot \frac{49 \cdot 50}{2} = 0.49
            \]
            \qed
        \end{solution}
        
        \part 
        Karl hat die Klausur bestanden. 
        Wie groß ist die Wahrscheinlichkeit, dass er vor der Klausur nicht mehr als 5 Aufgaben selbstständig bearbeitet hat?
        \begin{solution}
            Es gilt: 
            \[ 
                P(A_{\leq 5} \mid B) = \sum_{i=0}^5 \frac{P(A_i \cap B)}{P(B)} = \frac{1}{P(B)} \sum_{i=0}^5 P(A_i \cap B) = \frac{1}{P(B)} \cdot \frac{1}{50^2} \sum_{i=0}^5 i = \frac{15}{0.49 \cdot 2500} \approx 0.01224 
            \]
            \qed
        \end{solution}
    \end{parts}
\end{questions}
\end{document}