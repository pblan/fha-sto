
\documentclass{abgabe}
\begin{document}

\begin{questions}
    \question
    Bei  einer  Untersuchungsmethode  auf  Lungentuberkulose  wird  ein  Kranker  mit 90\%iger Sicherheit  als  krank  und  ein  Gesunder  mit 99\%iger  Sicherheit  als  gesund  erkannt.  
    In der  Bevölkerung  gibt  es 0.1\% Kranke.  
    Wie  groß  ist  unter  diesen  Voraussetzungen  die Wahrscheinlichkeit dafür, dass    
    \begin{parts}
        \part
        eine untersuchte Person dieser Bevölkerung als krank eingestuft wird?
        \begin{solution}
            Seien:
            \begin{itemize}
                \item $K := \{ \text{Eine Person ist krank} \}$
                \item $E := \{ \text{Eine Person wurde als gesund eingestuft} \}$
            \end{itemize}
            
            \begin{center}
                \begin{forest}
                    [, wide
                            [
                                $K$, above left prob={$0.1\%$}, wide 
                                    [
                                        $E$, above left prob={$10\%$}, wide
                                    ]
                                    [
                                        $\conj{E}$, above right prob={$90\%$}, wide
                                    ]
                            ]
                            [
                                $\conj{K}$, above right prob={$99.9\%$}, wide 
                                    [
                                        $E$, above left prob={$99\%$}, wide
                                    ]
                                    [
                                        $\conj{E}$, above right prob={$1\%$}, wide
                                    ]
                            ]
                    ]
                \end{forest}
            \end{center}
            
            Dann gilt: 
            \[ 
                P(\conj{E}) = P(E \mid K) + P(E \mid \conj{K}) = \frac{P(E \cap K)}{P(K)} + \frac{P(E \cap \conj{K})}{P(\conj{K})} = \ldots = 1.089\%
            \]
            \qed 
        \end{solution}
        
        \part 
        eine als krank eingestufte Person auch tatsächlich krank ist?
        \begin{solution}
            Es gilt: 
            \[ 
                P(K \mid \conj{E}) = \frac{P(K \cap \conj{E})}{P(\conj{E})} = \frac{0.1\% \cdot 90\%}{1.089\%} = 8.26\%
            \]
            \qed
        \end{solution}
        
        \part 
        eine als gesund eingestufte Person krank ist?
        \begin{solution}
            Es gilt: 
            \[ 
                P(K \mid E) = \frac{P(K \cap E)}{P(E)} = P(K \mid E) = \frac{P(K \cap E)}{1 - P(\conj{E})} = \frac{0.1\% \cdot 99\%}{1 - 1.089\%} \approx 0.01\% 
            \]
            \qed 
        \end{solution}
    \end{parts}
\end{questions}
\end{document}