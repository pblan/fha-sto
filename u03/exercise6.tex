
\documentclass{abgabe}
\begin{document}

\begin{questions}
    \question
    Bei  einer  Untersuchungsmethode  auf  Lungentuberkulose  wird  ein  Kranker  mit 90\%iger Sicherheit  als  krank  und  ein  Gesunder  mit 99\%iger  Sicherheit  als  gesund  erkannt.  
    In der  Bevölkerung  gibt  es 0.1\% Kranke.  
    Wie  groß  ist  unter  diesen  Voraussetzungen  die Wahrscheinlichkeit dafür, dass    
    \begin{parts}
        \part
        eine untersuchte Person dieser Bevölkerung als krank eingestuft wird?
        \begin{solution}

        \end{solution}
        
        \part 
        eine als krank eingestufte Person auch tatsächlich krank ist?
        \begin{solution}

        \end{solution}
        
        \part 
        eine als gesund eingestufte Person krank ist?
        \begin{solution}

        \end{solution}
    \end{parts}
\end{questions}
\end{document}