
\documentclass{abgabe}
\begin{document}

\begin{questions}
    \question
    Zwei unterscheidbare Würfel werden gleichzeitig geworfen und die Summe der beiden Augenzahlen betrachtet.
    \begin{parts}
        \part
        Bestimmen Sie die Ereignismenge der möglichen 2er Tupel (zwei Würfel), die eine gerade Augensumme bilden.
        \begin{solution}

            \qed
        \end{solution}
        
        \part 
        Berechnen Sie die Wahrscheinlichkeit eine gerade bzw. ungerade Augensumme zu würfeln
        \begin{solution}

            \qed 
        \end{solution}
    \end{parts}
    
    Im Anschluss wird mit den zwei Würfeln dreimal ein \gqq{Doppelwurf} ausgeführt. 
    Die Zufallsvariable $X$ bezeichne die Anzahl der insgesamt geraden Augensumme.
    
    \begin{parts}
        \setcounter{partno}{2}
        \part 
        Bestimmen Sie von der Zufallsvariable $X$
        \begin{subparts}
            \subpart 
            die Wahrscheinlichkeitsfunktion
            \begin{solution}

                \qed
            \end{solution}
            
            \subpart 
            die Verteilungsfunktion
            \begin{solution}

                \qed
            \end{solution}
        \end{subparts}
        
        \part 
        Stellen Sie die Funktionen grafisch dar (Stabdiagramm und Verteilungsfunktion).
        \begin{solution}

            \qed
        \end{solution}
    \end{parts}
\end{questions}
\end{document}