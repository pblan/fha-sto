
\documentclass{abgabe}
\begin{document}

\begin{questions}
    \question
    Zwei unterscheidbare Würfel werden gleichzeitig geworfen und die Summe der beiden Augenzahlen betrachtet.
    \begin{parts}
        \part
        Bestimmen Sie die Ereignismenge der möglichen 2er Tupel (zwei Würfel), die eine gerade Augensumme bilden.
        \begin{solution}
            Sei $E := \{ \text{Mit zwei Würfeln wird eine gerade Augensumme geworfen} \}$. 
            
            Es gilt: 
            \[ 
                \begin{aligned}
                    E = 
                    \{ 
                     & (1,1), (1,3), (1,5), (2,2), (2,4), (2,6), \\
                     & (3,1), (3,3), (3,5), (4,2), (4,4), (4,6), \\
                     & (5,1), (5,3), (5,5), (6,2), (6,4), (6,6) 
                    \}
                \end{aligned}
            \]
            \qed
        \end{solution}
        
        \part 
        Berechnen Sie die Wahrscheinlichkeit eine gerade bzw. ungerade Augensumme zu würfeln
        \begin{solution}
            Sei $O := \{ \text{Mit zwei Würfeln wird eine ungerade Augensumme geworfen} \}$. 
            
            Es gilt: 
            \[ 
                O = \conj{E} \quad \land \quad \abs{E} = \abs{O} \quad \land \quad  E \cup O = \Omega \quad \implies P(E) = P(O) = \frac{1}{2}
            \]
            \qed 
        \end{solution}
    \end{parts}
    
    \newpage
    
    Im Anschluss wird mit den zwei Würfeln dreimal ein \gqq{Doppelwurf} ausgeführt. 
    Die Zufallsvariable $X$ bezeichne die Anzahl der insgesamt geraden Augensummen.
    
    \begin{parts}
        \setcounter{partno}{2}
        \part 
        Bestimmen Sie von der Zufallsvariable $X$
        \begin{subparts}
            \subpart 
            die Wahrscheinlichkeitsfunktion
            \begin{solution}
                Die Wahrscheinlichkeitsverteilung der \emph{diskreten} Zufallsvariablen $X$ lässt sich durch die \emph{Wahrscheinlichkeitsfunktion}
                \[ 
                    P(X = x) = f(x) = 
                    \begin{cases}
                        \nicefrac{1}{8} & \text{für} \ x = 0 \\
                        \nicefrac{3}{8} & \text{für} \ x = 1 \\
                        \nicefrac{3}{8} & \text{für} \ x = 2 \\
                        \nicefrac{1}{8} & \text{für} \ x = 3 \\
                        0               & \text{sonst}
                    \end{cases}
                \]
                beschreiben.
                \qed
            \end{solution}
            
            \subpart 
            die Verteilungsfunktion
            \begin{solution}
                Die Wahrscheinlichkeitsverteilung der \emph{diskreten} Zufallsvariablen $X$ lässt sich durch die \emph{Verteilungsfunktion}
                \[ 
                    P(X \leq x) = F(x) = 
                    \begin{cases}
                        0               & \text{für} \ x < 0        \\
                        \nicefrac{1}{8} & \text{für} \ 0 \leq x < 1 \\
                        \nicefrac{1}{2} & \text{für} \ 1 \leq x < 2 \\
                        \nicefrac{7}{8} & \text{für} \ 2 \leq x < 3 \\
                        1               & \text{für} \ x \geq 3 
                    \end{cases}
                \]
                beschreiben.
                \qed
            \end{solution}
        \end{subparts}
        
        \newpage
        \part 
        Stellen Sie die Funktionen grafisch dar (Stabdiagramm und Verteilungsfunktion).
        \begin{solution}
            \begin{center}
                \includegraphics[width=.6\linewidth]{exercise1_bar_chart}
                \includegraphics[width=.6\linewidth]{exercise1_cfd}
            \end{center}
            \qed
        \end{solution}
    \end{parts}
\end{questions}
\end{document}