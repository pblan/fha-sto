
\documentclass{abgabe}
\begin{document}

\begin{questions}
    \question
    Die Dichtefunktion einer stetigen Verteilung laute
    \[ 
        f(x) = 
        \begin{cases}
            ax^2(3-x) & \text{für} \ 0 \leq x \leq 3 \\ 
            0         & \text{sonst}
        \end{cases}
    \]
    \begin{parts}
        \part
        Bestimmen Sie den Parameter $a$.
        \begin{solution}
            Eine Wahrscheinlichkeitsfunktion $f(x)$ muss \emph{normiert} sein, das heißt: 
            \[ 
                \int^\infty_{-\infty} f(t) \diff t = 1 \qquad \left(\impliedby \lim_{x \to \infty} F(x) = 1\right)
            \]
            
            Es gilt: 
            \begin{alignat*}{2}
                             & \int_0^3 f(t) \diff t                                                 &  & = 1            \\ 
                \equiv \quad & \int_0^3 at^2(3-t) \diff t                                            &  & = 1            \\ 
                \equiv \quad & a\left( 3\int_0^3 t^2 \diff t - \int_0^3 t^3 \diff t\right)           &  & = 1            \\ 
                \equiv \quad & 3\int_0^3 t^2 \diff t - \int_0^3 t^3 \diff t                          &  & = \frac{1}{a}  \\ 
                \equiv \quad & 3 \left[ \frac{t^3}{3} \right]_0^3 - \left[ \frac{t^4}{4} \right]_0^3 &  & = \frac{1}{a}  \\ 
                \equiv \quad & 27 - \frac{81}{4}                                                     &  & = \frac{1}{a}  \\ 
                \equiv \quad & \frac{27}{4}                                                          &  & = \frac{1}{a}  \\ 
                \equiv \quad & a                                                                     &  & = \frac{4}{27}
            \end{alignat*}
            \qed
        \end{solution}
        
        \part 
        Wie lautet die zugehörige Verteilungsfunktion?
        \begin{solution}
            Es gilt:
            \[ 
                P(X \leq x) = \int_{-\infty}^{\infty} f(t) \diff t = \int_{0}^{3} f(t) \diff t = \int_{0}^{3} \frac{4}{27} \cdot t^2(3-t) \diff t = \frac{4}{27} \left( t^3 - \frac{t^4}{4} \right)
            \]
            
            Und damit:
            \[ 
                P(X \leq x) = F(x) = 
                \begin{cases}
                    0                                                       & \text{für} \ x < 0           \\
                    \nicefrac{4}{27} \left( x^3 - \nicefrac{x^4}{4} \right) & \text{für} \ 0 \leq x \leq 3 \\
                    1                                                       & \text{für} \ x > 3 
                \end{cases}
            \]
            \qed
        \end{solution}
        
        \newpage
        \part 
        Berechnen Sie die Wahrscheinlichkeit, dass die Zufallsvariable $X$ einen Wert kleiner oder gleich 2 annimmt
        \begin{subparts}
            \subpart 
            über die Dichtefunktion
            \begin{solution}
                Es gilt: 
                \[ 
                    P(X \leq 2) = F(2) = \frac{4}{27}\left( 2^3 - \frac{2^4}{4} \right) = \frac{16}{27}
                \]
                \qed
            \end{solution}
            
            \subpart 
            über die Verteilungsfunktion
            \begin{solution}
                Es gilt: 
                \[ 
                    P(X \leq 2) = \int_{-\infty}^2 f(t) \diff t = \int_{0}^2 f(t) \diff t = \left[ \frac{4}{27} \left( x^3 - \frac{x^4}{4} \right) \right]_{0}^2 = \frac{16}{27}
                \]
                \qed
            \end{solution}
        \end{subparts}
    \end{parts}
\end{questions}
\end{document}