
\documentclass{abgabe}
\begin{document}

\begin{questions}
    \question
    Sei $X$ eine Zufallsvariable mit einer stetigen Verteilungsfunktion $F(x)$ der Form
    \[ 
        F(x) = 
        \begin{cases}
            0                         & \text{für} \ x < -2           \\ 
            \frac{1}{4} + \frac{x}{8} & \text{für} \ -2 \leq x \leq 0 \\ 
            c_1 + c_2 (1-e^{-x})      & \text{für} \ x > 0            \\ 
        \end{cases}
    \]
    \begin{parts}
        \part
        Bestimmen Sie die Konstanten $c_1$ und $c_2$.
        \begin{solution}
            Die Verteilungsfunktion muss normiert sein. 
            Es gilt: 
            \begin{alignat*}{2}
                             & \lim_{x\to \infty} F(x)                                &  & = 1                       \\  
                \equiv \quad & \lim_{x\to \infty} \left( c_1 + c_2 (1-e^{-x}) \right) &  & = 1                       \\  
                \equiv \quad & \lim_{x\to \infty} c_2 (1-e^{-x})                      &  & = 1 - c_1                 \\  
                \equiv \quad & \lim_{x\to \infty} (1-e^{-x})                          &  & = \frac{1 - c_1}{c_2}     \\  
                \equiv \quad & \lim_{x\to \infty} -e^{-x}                             &  & = \frac{1 - c_1}{c_2} - 1 \\  
                \equiv \quad & \lim_{x\to \infty} e^{-x}                              &  & = \frac{c_1 - 1}{c_2} + 1 \\  
                \equiv \quad & 0                                                      &  & = \frac{c_1 - 1}{c_2} + 1 \\  
                \equiv \quad & c_2                                                    &  & = 1 - c_1                 \\  
            \end{alignat*}
            
            Weiterhin muss die Verteilungsfunktion rechtsstetig sein.
            Es gilt damit: 
            \begin{alignat*}{2}
                               & \lim_{x\downarrow 0} F(x)                 &  & = F(0)        \\ 
                \equiv \quad   & \lim_{x\downarrow 0} c_1 + c_2 (1-e^{-x}) &  & = \frac{1}{4} \\ 
                \equiv \quad   & c_1 + (1 - c_1) (1-1)                     &  & = \frac{1}{4} \\ 
                \equiv \quad   & c_1                                       &  & = \frac{1}{4} \\ 
                \implies \quad & c_2                                       &  & = \frac{3}{4}
            \end{alignat*}
            \qed
        \end{solution}
        
        \part 
        Berechnen Sie den Erwartungswert $E(X)$.
        \begin{solution}

            \qed 
        \end{solution}
        
        \part 
        Berechnen Sie die Wahrscheinlichkeit, dass $X$ mindestens den Wert 2 annimmt, wenn man weiß, dass $X$ positiv ist.
        \begin{solution}

            \qed 
        \end{solution}
        
    \end{parts}
\end{questions}
\end{document}