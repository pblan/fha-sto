
\documentclass{abgabe}
\begin{document}

\begin{questions}
    \question
    Die \gqq{Intaktwahrscheinlichkeiten} (Wahrscheinlichkeit, dass eine Anlage, Baugruppe, Bauelement etc. wie vorgesehen arbeitet), bezogen auf ein festes Zeitintervall, betragen für zwei unabhängig voneinander arbeitende Anlagen $0.9$ bzw. $0.8$. 
    Die Zufallsgröße $X$ sei die zufällige Anzahl der in einem solchen Zeitintervall intakten Anlagen. 
    Bestimmen Sie
    \begin{parts}
        \part
        die Verteilungstabelle von $X$ und das entsprechende Stabdiagramm,
        \begin{solution}

            \qed
        \end{solution}
        
        \part 
        die Wahrscheinlichkeit dafür, dass wenigstens eine Anlage intakt ist,
        \begin{solution}

            \qed
        \end{solution}
        
        \part 
        die Verteilungsfunktion von $X$ mit einer grafischen Darstellung.
        \begin{solution}

            \qed
        \end{solution}
    \end{parts}
\end{questions}
\end{document}