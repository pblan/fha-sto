
\documentclass{abgabe}
\begin{document}

\begin{questions}
    \question
    Bestimmen Sie die Verteilungsfunktion $F_y(y)$ und die Dichtefunktion $f_y(y)$ für die transformierte Zufallsvariable $Y$, die sich als $Y = g(X)$ aus der ursprünglichen Zufallsvariablen $X$ mit bekannter Verteilungsfunktion $F_x(x)$ und bekannter Dichtefunktion $f_x(x)$ ergibt: 
    \begin{parts}
        \part
        $g(X) = aX + b \quad a, b \in \R \quad a \neq 0$
        \begin{solution}
            Wir wissen: 
            \begin{alignat*}{1}
                F_Y(y) & = P(Y \leq y)      \\ 
                       & = P(g(X) \leq y)   \\ 
                       & = P(aX + b \leq y) \\ 
                       & = P(aX \leq y-b)   \\ 
                       & = 
                \begin{cases}
                    P(X \leq \frac{y - b}{a}) & \text{für} \ a > 0 \\
                    P(X \geq \frac{y - b}{a}) & \text{für} \ a < 0
                \end{cases}   \\ 
                       & = 
                \begin{cases}
                    P(X \leq \frac{y - b}{a})  & \text{für} \ a > 0 \\
                    1 - P(X < \frac{y - b}{a}) & \text{für} \ a < 0
                \end{cases}   \\ 
                       & = 
                \begin{cases}
                    F_X(\frac{y - b}{a})     & \text{für} \ a > 0 \\
                    1 - F_X(\frac{y - b}{a}) & \text{für} \ a < 0
                \end{cases}
            \end{alignat*}
            
            Und damit: 
            \begin{alignat*}{1}
                f_Y(y) & = 
                \begin{cases}
                    \frac{\diff}{\diff y} F_X(\frac{y - b}{a})                    & \text{für} \ a > 0 \\
                    \frac{\diff}{\diff y} \left( 1 - F_X(\frac{y - b}{a}) \right) & \text{für} \ a < 0
                \end{cases}                                        \\
                       & = 
                \begin{cases}
                    f_X(\frac{y - b}{a}) \cdot \frac{1}{a}   & \text{für} \ a > 0 \\
                    - f_X(\frac{y - b}{a}) \cdot \frac{1}{a} & \text{für} \ a < 0
                \end{cases}                                        \\ 
                       & = f_X\left(\frac{y-b}{a}\right) \cdot \frac{1}{\abs{a}}
            \end{alignat*}
            \qed
        \end{solution}
        
        \newpage
        \part 
        $g(X) = 3X - 1 \quad \text{mit} \ f_X(x) =
            \begin{cases}
                \nicefrac{4}{27}(3x^2 - x^3) & \text{für} \ 0 \leq x \leq 3 \\    
                0                            & \text{sonst}
            \end{cases}$
        \begin{solution}
            Es gilt: 
            \[
                f_Y(y) = f_X\left(\frac{y+1}{3}\right) \cdot \frac{1}{\abs{3}} = 
                \begin{cases}
                    \frac{1}{3} \cdot \frac{4}{27} \left( 3\left( \frac{y+1}{3} \right)^2 - \left( \frac{y+1}{3} \right)^3 \right) & \text{für}\footnotemark[1] \ -1 \leq y \leq 8 \\ 
                    0                                                                                                              & \text{sonst}
                \end{cases}
            \]
            
            Und damit: 
            \[ 
                F_Y(y) = F_X\left( \frac{y+1}{3} \right) = 
                \begin{cases}
                    0                                                                                    & \text{für} \ y < -1           \\    
                    \frac{4}{27} \left(\frac{y+1}{3}\right)^3 - \frac{1}{27}\left(\frac{y+1}{3}\right)^4 & \text{für} \ -1 \leq y \leq 8 \\    
                    1                                                                                    & \text{für} \ y > 8     
                \end{cases}
            \]
            \qed 
        \end{solution}
        
        \footnotetext[1]{$0 \leq x \leq 3 \quad \land \quad y = g(x) = 3x - 1 \quad \implies \quad -1 \leq y \leq 8$}
    \end{parts}
\end{questions}
\end{document}