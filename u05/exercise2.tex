
\documentclass{abgabe}
\begin{document}

\begin{questions}
    \question
    Gegeben ist ein Spannungssignal $X$ mit Gaußscher Dichtefunktion $f_X(x)$, Erwartungswert $\mu_x = 1\si{V}$ und Varianz $\sigma_X^2 = 0.25\si{\volt}^2$. 
    Das Signal wird durch die Funktion $Y = g(X) = 2X + 1.5\si{\V}$ in ein Ausgangssignal $Y$ transformiert.
    
    Bestimmen Sie den Erwartungswert $\mu_Y$ sowie die Varianz $\sigma^2_Y$ des Ausgangssignals.
    \begin{solution}
        Es handelt sich hier im eine \emph{lineare} Transformation. 
        
        Damit gilt: 
        \[ 
            \Mean(Y) = \Mean(2X + 1.5\si{\V}) = 2\Mean(X) + 1.5\si{\V} = 2\mu_X + 1.5\si{\V} = 2 \cdot 1\si{\V} + 1.5\si{\V} = 3.5\si{\V}
        \]
        \[
            \Var(Y) = \Var(2X + 1.5\si{\V}) = 2^2 \Var(X) = 4 \sigma_X^2 = 4 \cdot 0.25\si{\volt^2} = 1\si{\V^2}
        \]
        \qed
    \end{solution}
\end{questions}
\end{document}