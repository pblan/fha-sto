
\documentclass{abgabe}
\begin{document}

\begin{questions}
    \question
    Berechnen Sie
    \begin{parts}
        \part
        den Erwartungswert,
        \begin{solution}
            Es gilt: 
            \[ 
                \Mean(X) = \sum^{4}_{i=0} x_i \cdot P(X = x_i) = (-2) \cdot \frac{1}{4} + 2 \cdot \frac{1}{6} + 4 \cdot \frac{1}{4} + 6 \cdot \frac{1}{4} + 8 \cdot \frac{1}{12} = 3
            \]
            \qed
        \end{solution}
        
        \part 
        die Varianz und
        \begin{solution}
            Es gilt: 
            \[ 
                \Var(X) = \sum^{4}_{i=0} (x_i - \mu)^2 \cdot P(X = x_i) = (-5)^2 \cdot \frac{1}{4} + (-1)^2 \cdot \frac{1}{6} + (-1)^2 \cdot \frac{1}{4} + 3^2 \cdot \frac{1}{4} + 5^2 \cdot \frac{1}{12} = 11
            \]
            \qed 
        \end{solution}
        
        \part 
        die Standardabweichung
        \begin{solution}
            Es gilt: 
            \[ 
                \sigma = \sqrt{\Var(X)} = \sqrt{11}
            \]
            \qed 
        \end{solution}
    \end{parts}
    der folgenden diskreten Verteilung: 
    
    \begin{center}        
        \begin{tabular}{|C|C|C|C|C|C|}
            \hline
            x_i        & -2              & 2               & 4               & 6               & 8                \\
            \hline
            P(X = x_i) & \nicefrac{1}{4} & \nicefrac{1}{6} & \nicefrac{1}{4} & \nicefrac{1}{4} & \nicefrac{1}{12} \\
            \hline
        \end{tabular}
    \end{center}
\end{questions}
\end{document}