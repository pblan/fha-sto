
\documentclass{abgabe}
\begin{document}

\begin{questions}
    \question
    Gegeben ist die Dichtefunktion der Zufallsvariablen $X$ als 
    \[ 
        f_X(x) = 
        \begin{cases}
            \frac{a}{1+x^2} & \text{für} \ \abs{x} \leq 1   \\    
            0               & \text{für} \ \abs{x} > 1     
        \end{cases}
    \]
    \begin{parts}
        \part
        Wie groß ist $a$?
        \begin{solution}
            Es gilt: 
            \begin{alignat*}{2}
                             & \int^{\infty}_{-\infty} f_X(x) \diff x                        &  & = 1             \\ 
                \equiv \quad & \int^{-1}_{1} \frac{a}{1+x^2} \diff x                         &  & = 1             \\ 
                \equiv \quad & a \int^{-1}_{1} \frac{1}{1+x^2} \diff x                       &  & = 1             \\ 
                \equiv \quad & a \left[ \arctan(x) \right]^{-1}_{1}                          &  & = 1             \\ 
                \equiv \quad & a \left( \frac{\pi}{4} - \left( \frac{\pi}{4} \right) \right) &  & = 1             \\ 
                \equiv \quad & \frac{a\pi}{2}                                                &  & = 1             \\ 
                \equiv \quad & a                                                             &  & = \frac{2}{\pi}
            \end{alignat*}
            \qed
        \end{solution}
        
        \part 
        Wie groß ist der Erwartungswert $E(X)$ der Zufallsvariablen $X$?
        \begin{solution}
            Es gilt: 
            \begin{alignat*}{1}
                \Mean(X) & = \int^{\infty}_{-\infty} x f(x) \diff x                          \\  
                         & = \frac{\pi}{2} \int^{1}_{-1} \frac{x}{1 + x^2} \diff x           \\ 
                         & = \footnotemark \frac{\pi}{4} \int^{1}_{x=-1} \frac{1}{u} \diff u \\ 
                         & = \frac{\pi}{4} \left[ \ln(u) \right]^{1}_{x=-1}                  \\ 
                         & = \frac{\pi}{4} \left[ \ln(1+x^2) \right]^{1}_{-1}                \\ 
                         & = \frac{\pi}{4} \left( \ln(2) - \ln(2) \right)                    \\ 
                         & = 0 
            \end{alignat*}
            
            \qed 
        \end{solution}
        \footnotetext{$u := 1+x^2 \implies \frac{\diff u}{\diff x} = 2x \iff \diff x = \frac{\diff u}{2x}$}
        
        
        \newpage
        \part 
        Berechnen Sie die Varianz $\Var(X)$ der Zufallsvariablen $X$.
        \begin{solution}
            Es gilt: 
            \begin{alignat*}{1}
                \Var(X) & = \int^{\infty}_{-\infty} (x - \mu)^2 f_X(x) \diff x                                 \\  
                        & = \frac{\pi}{2} \int^{1}_{-1} (x - 0)^2 \frac{1}{1 + x^2} \diff x                    \\ 
                        & = \frac{\pi}{2} \int^{1}_{-1} \frac{x^2}{1 + x^2} \diff x                            \\ 
                        & = \frac{\pi}{2} \int^{1}_{-1} \frac{1 + x^2 - 1}{1 + x^2} \diff x                    \\ 
                        & = \frac{\pi}{2} \int^{1}_{-1} 1 - \frac{1}{1 + x^2} \diff x                          \\ 
                        & = \frac{\pi}{2} \left[ x - \arctan(x) \right]^{1}_{-1}                               \\ 
                        & = \frac{\pi}{2} \left( 1 - \frac{\pi}{4} - \left( -1 + \frac{\pi}{4} \right) \right) \\ 
                        & = 2 - \frac{\pi}{2}
            \end{alignat*}
            \qed 
        \end{solution}
    \end{parts}
\end{questions}
\end{document}