
\documentclass{abgabe}
\begin{document}

\begin{questions}
    \question
    Aus einem Skatspiel mit 32 Karten wird eine Karte zufällig entnommen.
    \begin{parts}
        \part 
        Sie spielen folgende Spielvariante: 
        Jede Karte wird einzeln gezogen. 
        Sie notieren, welche Karte es gewesen ist und legen die Karte zurück. 
        Sie wiederholen dieses Vorgehen 10-mal. 
        Wie wahrscheinlich ist es, dass genau zwei Buben dabei gewesen sind?
        \begin{solution}
            Es handelt sich offenbar um ein \emph{Bernoulli}-Experiment. 
            Damit haben wir eine \emph{Binomialverteilung} gegeben mit
            \begin{itemize}
                \item $n = 10$,
                \item $p = \nicefrac{4}{32} = \nicefrac{1}{8}$.
            \end{itemize}
            
            Damit gilt für $x \in \interval{0,5}_{\N_0}$: 
            \[ 
                b(x;n,p) = b\left( x;10, \nicefrac{1}{8} \right) = f(x) = P(X = x) = \binom{n}{x} \cdot p^x \cdot q^{n-x} = \binom{10}{x} \cdot \left(\frac{1}{8}\right)^x \cdot \left(\frac{7}{8}\right)^{10-x}
            \]
            
            Und damit: 
            \[ 
                b(2;10,\nicefrac{1}{8}) = \binom{10}{2} \cdot \left(\frac{1}{8}\right)^2 \cdot \left(\frac{7}{8}\right)^{8} \approx 24.16\%
            \]
            \qed
        \end{solution}
        
        \newpage
        \part 
        Wie oft muss man eine Karte ziehen, damit die Wahrscheinlichkeit dafür, mindestens ein rotes Ass zu ziehen, größer als $0.5$ wird?
        \begin{solution}
            Für die Wahrscheinlichkeit dafür, dass mindestens ein rotes Ass gezogen wird, gilt: 
            \[ 
                P(X \geq 1) = 1 - P(X < 1) = 1 - P(X = 0)
            \]
            Damit gilt mit 
            \begin{itemize}
                \item $n = n$,
                \item $p = \nicefrac{1}{16}$:
            \end{itemize}
            \[ 
                b(x;n,p) = b(x,n,\nicefrac{1}{16}) = \binom{n}{x} \cdot p^x \cdot q^{n-x} = \binom{n}{x} \cdot \left(\frac{1}{16}\right)^x \cdot \left(\frac{15}{16}\right)^{n-x}
            \]
            
            Mit $1 - P(X = 0) \geq 0$ gilt: 
            \begin{alignat*}{2}
                               & 1 - P(X = 0)                                                                              &  & \geq 0.5                                   \\ 
                \implies \quad & 1 - \binom{n}{0} \cdot \left(\frac{1}{16}\right)^0 \cdot \left(\frac{15}{16}\right)^{n-0} &  & \geq 0.5                                   \\
                \implies \quad & 1 - 1 \cdot 1 \cdot \left(\frac{15}{16}\right)^{n}                                        &  & \geq 0.5                                   \\
                \implies \quad & \left(\frac{15}{16}\right)^{n}                                                            &  & \leq 0.5                                   \\
                \implies \quad & n \cdot \ln \frac{15}{16}                                                                &  & \leq \ln \frac{1}{2}                      \\
                \implies \quad & n                                                                                         &  & \geq \frac{\ln \nicefrac{1}{2}}{\ln \nicefrac{16}{15}} \\
                \implies \quad & n                                                                                         &  & \geq \frac{\ln 1 - \ln 2}{\ln 16 - \ln 15} \\
                \implies \quad & n                                                                                         &  & \geq 10.74                               \\
            \end{alignat*}
            
            Damit muss man mindestens 11-mal ziehen. \qed
        \end{solution}
    \end{parts}
\end{questions}
\end{document}