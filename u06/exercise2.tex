
\documentclass{abgabe}
\begin{document}

\begin{questions}
    \question
    Eine Firma liefert Dichtungen in Packungen zu 100 Stück. 
    Eine Packung darf laut Liefervertrag 10\% Ausschuss enthalten. 
    Jede Packung wird geprüft, indem man 10 Stück zufällig und ohne Zurücklegen entnimmt. 
    Sind diese 10 Stück alle einwandfrei, wird die Packung angenommen. 
    Anderenfalls wird sie zurückgewiesen.
    Wie groß ist bei diesem Prüfverfahren die Wahrscheinlichkeit ungerechtfertigter Reklamationen, indem eine Packung zurückgewiesen wird, obwohl sie gerade noch den Lieferbedingungen entspricht?
    \begin{solution}
        Offensichtlich gilt: 
        \[ 
            X = \{ \text{Anzahl der defekten Dichtungen} \} \sim \hdist(x;N,M,n) = \hdist(x;100,10,10)
        \] 

        Und damit: 
        \[ 
            P(X = x) = \frac{\binom{M}{x} \cdot \binom{N-M}{n-x}}{\binom{N}{n}} = \frac{\binom{10}{x} \cdot \binom{90}{10-x}}{\binom{100}{10}}
        \] 

        Die Packung wird genau dann fälschlicherweise zurückgewiesen, wenn mindestens ein Stück der Stichprobe fehlerhaft ist.
        Damit gilt: 
        \[ 
        P(X \geq 1) = 1 - P(X < 1) = 1 - P(X = 0) = 1 - \frac{\binom{10}{0} \cdot \binom{90}{10}}{\binom{100}{10}} \approx 66.95\%
        \]
        \qed
    \end{solution}
\end{questions}
\end{document}