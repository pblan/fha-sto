
\documentclass{abgabe}
\begin{document}

\begin{questions}
    \question
    Betrachten Sie folgendes Zufallsexperiment:
    Eine faire Münze mit den Seiten $0$ und $1$ wird zweimal unabhängig geworfen.
    Die Zufallsvariable $Z_1$ bezeichne das Ergebnis des ersten Wurfes, entsprechend $Z_2$ das des zweiten Wurfes.
    \begin{parts}
        \part 
        Berechnen Sie die Wahrscheinlichkeit aller möglichen 2er Tupel, die bei dem Doppelmünzwurf entstehen.
        \begin{solution}
            Es gilt: 
            \[ 
                \Omega = \{ (Z_1, Z_2) \mid Z_1, Z_2 \in \{ 0;1 \} \} \quad \text{mit} \quad \abs{\Omega} = 4
            \]
            sowie: 
            \[ 
                P(Z_1 = 0) = P(Z_1 = 1) = P(Z_2 = 0) = P(Z_2 = 1) = \frac{1}{2}
            \]
            und da die beiden Zufallsvariablen offensichtlich stochastisch unabhängig voneinander sind:
            \[ 
                P(Z_1;Z_2) = P(Z_1) \cdot P(Z_2) = \frac{1}{4}
            \]
            \qed
        \end{solution}
    \end{parts}
    
    Wir betrachten nun die neuen Zufallsvariablen
    \[ 
        X = Z_1 - Z_2 \quad \text{und} \quad Y = Z_1 + Z_2
    \]
    \begin{parts}
        \part 
        Welche Werte haben die beiden Zufallsvariablen?
        \begin{solution}
            Es gilt: 
            \begin{center}
                \begin{tabular}{|C|C|C|C|C|}
                    \hline 
                    Z_1 & Z_2 & P(Z_1;Z_2)      & X = Z_1 - Z_2 & Y = Z_1 + Z_2 \\ 
                    \hline 
                    0   & 0   & \nicefrac{1}{4} & 0             & 0             \\
                    0   & 1   & \nicefrac{1}{4} & -1            & 1             \\
                    1   & 0   & \nicefrac{1}{4} & 1             & 1             \\
                    1   & 1   & \nicefrac{1}{4} & 0             & 2             \\
                    \hline 
                \end{tabular}
            \end{center}
            \qed
        \end{solution}
        
        \part 
        Bestimmen Sie die gemeinsame Verteilung von $X$ und $Y$ sowie die jeweiligen Randverteilungen (tabellarische Darstellung).
        \begin{solution}
            \begin{center}
                \begin{tabular}{|C|C|C|C|C|C|}
                    \hline 
                    $\diagbox{X}{Y}$ & 0               & 1               & 2               & f_1(x)          \\
                    \hline 
                    -1               & 0               & \nicefrac{1}{4} & 0               & \nicefrac{1}{4} \\
                    \hline 
                    0                & \nicefrac{1}{4} & 0               & \nicefrac{1}{4} & \nicefrac{1}{2} \\
                    \hline 
                    1                & 0               & \nicefrac{1}{4} & 0               & \nicefrac{1}{4} \\
                    \hline 
                    f_2(y)           & \nicefrac{1}{4} & \nicefrac{1}{2} & \nicefrac{1}{4} & 1               \\
                    \hline
                \end{tabular}
            \end{center}
            \qed
        \end{solution}
        
        \newpage 
        \part 
        Bestimmen Sie die Erwartungswerte von $X$ und $Y$. 
        \begin{solution}
            Es gilt offensichtlich: 
            \[ 
                E(Z_1) = E(Z_2) = \frac{1}{2}
            \]
            und damit 
            \[ 
                E(X) = E(Z_1 - Z_2) = E(Z_1) - E(Z_2) = \frac{1}{2} - \frac{1}{2} = 0
            \]
            \[
                E(Y) = E(Z_1 + Z_2) = E(Z_1) + E(Z_2) = \frac{1}{2} + \frac{1}{2} = 1  
            \]
            \qed
        \end{solution}
        
        \part 
        Berechnen Sie die Kovarianz von $X$ und $Y$.
        \begin{solution}
            Es gilt: 
            \begin{alignat*}{1}
                \Cov(X,Y) & = E(XY) - E(X)E(Y)                                                        \\
                          & = \sum_{i = 1}^3 \sum_{j = 1}^3 x_i \cdot y_j \cdot f(x_i;y_j) - E(X)E(Y) \\
                          & = -\frac{1}{4} + \frac{1}{4} - 0 \cdot 1                                  \\
                          & = 0
            \end{alignat*}
            \qed
        \end{solution}
        
        \part 
        Sind $X$ und $Y$ stochastisch unabhängig?
        \begin{solution}
            Es gilt: 
            \[ 
                f(-1;0) = 0 \neq \frac{1}{16} = \frac{1}{4} \cdot \frac{1}{4} = f_1(-1) \cdot f_2(0)
            \]
            
            Damit sind $X$ und $Y$ stochastisch abhängig.
            \qed
        \end{solution}
    \end{parts}
\end{questions}
\end{document}