
\documentclass{abgabe}
\begin{document}

\begin{questions}
    \question
    Gegeben sei die folgende zweidimensionale Wahrscheinlichkeitsfunktion
    \[ 
        f(x, y) = P (X = x, Y = y)
    \]
    \begin{center}
        \begin{tabular}{|C|C|C|C|}
            \hline 
            $\diagbox{X}{Y}$ & 0                & 1                & \\
            \hline
            1                & \nicefrac{1}{48} & \nicefrac{1}{16} & \\
            \hline
            2                & \nicefrac{1}{16} & \nicefrac{3}{16} & \\
            \hline
            3                & \nicefrac{5}{48} & \nicefrac{5}{16} & \\
            \hline
            3                & \nicefrac{1}{16} & \nicefrac{3}{16} & \\
            \hline
                             &                  &                  & \\
            \hline
        \end{tabular}
    \end{center}
    \begin{parts}
        \part 
        Leiten Sie die zweidimensionale Verteilungsfunktion ab.
        \begin{solution}

            \qed
        \end{solution}
        
        \part 
        Berechnen Sie die Randwahrscheinlichkeiten für beide Zufallsvariablen.
        \begin{solution}

            \qed
        \end{solution}
        
        \part 
        Überprüfen Sie, ob die Zufallsvariablen $X$ und $Y$ vollständig unabhängig sind.
        \begin{solution}

            \qed
        \end{solution}
        
        \part 
        Berechnen Sie aus den Randverteilungen die Erwartungswerte und Varianzen für $X$ und $Y$.
        \begin{solution}

            \qed
        \end{solution}
        
        \part 
        Berechnen Sie für die Kovarianz bzw. den Korrelationskoeffizienten der Zufallsvariablen $X$ und $Y$.
        \begin{solution}

            \qed
        \end{solution}
    \end{parts}
\end{questions}
\end{document}