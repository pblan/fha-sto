
\documentclass{abgabe}
\begin{document}

\begin{questions}
    \question
    Bei einem Messvorgang wird angenommen, dass er durch einen Zufallsvariable mit unbekannten Erwartungswert $\mu$ und einer Streuung $\sigma = 0.1$ [Maßeinheiten] angemessen beschrieben werden kann. 
    Bei einer Messreihe soll die Wahrscheinlichkeit, dass der Betrag der Differenz zwischen dem arithmetischen Mittel der Messwerte und $\mu$ kleiner als $0.02$ [Maßeinheiten] ist, mindestens 95\% sein. 
    Wie viele Messungen müssen Sie durchführen
    \begin{parts}
        \part 
        unter Anwendung der Ungleichung von Tschebyscheff?
        \begin{solution}

            \qed
        \end{solution}
        
        \newpage
        \part 
        unter Berücksichtigung, dass das arithmetische Mittel von $n$ unabhängigen Zufallsvariablen (für großes $n$) näherungsweise normalverteilt ist?
        \begin{solution}

            \qed
        \end{solution}
    \end{parts}
\end{questions}
\end{document}