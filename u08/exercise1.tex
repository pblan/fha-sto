
\documentclass{abgabe}
\begin{document}

\begin{questions}
    \question
    Bei einem Messvorgang wird angenommen, dass er durch einen Zufallsvariable mit unbekannten Erwartungswert $\mu$ und einer Streuung $\sigma = 0.1$ [Maßeinheiten] angemessen beschrieben werden kann. 
    Bei einer Messreihe soll die Wahrscheinlichkeit, dass der Betrag der Differenz zwischen dem arithmetischen Mittel der Messwerte und $\mu$ kleiner als $0.02$ [Maßeinheiten] ist, mindestens 95\% sein. 
    Wie viele Messungen müssen Sie durchführen
    \begin{parts}
        \part 
        unter Anwendung der Ungleichung von Tschebyscheff?
        \begin{solution}
            Wir wissen:
            \begin{itemize}
                \item $X = \{ \text{Messwerte einer Messung} \}$
                \item $\conj{X}_{(n)} = \{ \text{arithmetisches Mittel bei $n$ Messungen} \} = \frac{1}{n} \sum_{i = 1}^n X_i$
                \item $\mu = \ ?$
                \item $\sigma = 0.1$
                \item $\epsilon = 0.02$
                \item $P\left( \abs{\conj{X}_{(n)} - \mu} < 0.02 \right) \geq 95\%$
            \end{itemize}

            Nach Tschebyscheff gilt: 
            \[ 
                P \left( \abs{X - \mu} \geq \epsilon \right) \leq \frac{\sigma^2}{\epsilon^2} \quad \text{bzw.} \quad P \left( \abs{\conj{X}_{(n)} - \mu} \geq \epsilon \right) \leq \frac{\sigma^2}{n\epsilon^2}
            \] 

            Damit gilt: 
            \begin{alignat*}{2}
                & P \left( \abs{\conj{X}_{(n)} - \mu} \geq \epsilon \right) && \leq \frac{\sigma^2}{n\epsilon^2}  \\
                \equiv \quad & 1 - P \left( \abs{\conj{X}_{(n)} - \mu} < \epsilon \right) && \leq \frac{\sigma^2}{n\epsilon^2}  \\
                \equiv \quad & P \left( \abs{\conj{X}_{(n)} - \mu} < \epsilon \right) && \geq 1 - \frac{\sigma^2}{n\epsilon^2} \\
                \implies \quad & 1 - \frac{0.1^2}{n \cdot 0.02^2} && \geq 0.95 \\
                \equiv \quad & \frac{0.1^2}{n \cdot 0.02^2} && \leq 0.05 \\
                \equiv \quad & n && \geq 500
            \end{alignat*}
            \qed
        \end{solution}
        
        \part 
        unter Berücksichtigung, dass das arithmetische Mittel von $n$ unabhängigen Zufallsvariablen (für großes $n$) näherungsweise normalverteilt ist?
        \begin{solution}
            Es gilt: 
            \[ 
                \conj{X}_{(n)} \sim N(\mu_{\conj{X}_{(n)}}, \sigma^2_{\conj{X}_{(n)}})
            \] 
            \qed
        \end{solution}
    \end{parts}
\end{questions}
\end{document}