
\documentclass{abgabe}
\begin{document}

\begin{questions}
    \question
    Der zufällige Fehler X eines Messgerätes habe den Erwartungswert $\Mean(X) = 0 \si{\um}$ und die Standardabweichung $\sigma = 20 \si{\um}$. 
    Damit liegen keine systematischen Messfehler vor, es können nur zufällige Messfehler auftreten. 
    Gesucht ist die Wahrscheinlichkeit, dass das arithmetische Mittel aus 25 unabhängigen Messungen von der wahren Länge des zu messenden Werkstücks dem Betrag nach um höchstens $3 \si{\um}$ abweicht.
    \begin{solution}

        \qed
    \end{solution}
\end{questions}
\end{document}

