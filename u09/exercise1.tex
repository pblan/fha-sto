
\documentclass{abgabe}
\begin{document}

\begin{questions}
    \question
    An der Scanner-Kasse eines Supermarktes wurden für 50 aufeinanderfolgende Kunden folgende Bedienungszeit [in Sekunden] registriert:
    
    \begin{center}
        \begin{tabular}{|C|C|C|C|C|C|C|C|C|C|}
            \hline
            15 & 18 & 18 & 19 & 19 & 20 & 22 & 22 & 22 & 22 \\ 
            \hline
            23 & 24 & 27 & 28 & 29 & 31 & 32 & 33 & 36 & 37 \\ 
            \hline
            37 & 38 & 38 & 39 & 39 & 39 & 40 & 40 & 40 & 41 \\ 
            \hline
            41 & 42 & 42 & 43 & 44 & 48 & 49 & 49 & 50 & 51 \\ 
            \hline
            51 & 51 & 52 & 53 & 54 & 57 & 58 & 62 & 64 & 68 \\ 
            \hline
        \end{tabular}
    \end{center}
    \begin{parts}
        \part 
        Bestimmen Sie
        \begin{subparts}
            \subpart 
            den Modalwert,
            \begin{solution}
                \[ 
                    \Modal{x} = 22
                \]
                \qed
            \end{solution}
            \subpart 
            den Median,
            \begin{solution}
                \[ 
                    \tilde{x} = x_{\nicefrac{1}{2}} = 39
                \]
                \qed
            \end{solution}
            \subpart 
            das obere und untere Quartil,
            \begin{solution}
                \[ 
                    x_{\nicefrac{1}{4}} = 27 \quad \land \quad x_{\nicefrac{3}{4}} = 50
                \]
                \qed
            \end{solution}
            \subpart 
            das arithmetische Mittel sowie
            \begin{solution}
                \[ 
                    \conj{x} = \frac{1}{50} \sum_{i=1}^{50} x_i = \ldots = \frac{1917}{50} = 38.34
                \]
                \qed
            \end{solution}
            \subpart 
            die empirische Standardabweichung
            \begin{solution}
                \[ 
                    \conj{s}^2 = \sum_{j=1}^k \left( a_j - \conj{x} \right)^2 f_j = \sum_{j=1}^k \left( a_j - \frac{444}{25} \right)^2 f_j = \ldots = 606.0733
                \]
                \qed
            \end{solution}
            der Bedienungszeit.
        \end{subparts}
        
        \part 
        Erstellen Sie ein Histogramm unter Verwendung der Klassengrenzen
        \[ 
            0,20,30,40,50,70
        \]
        wobei die Klassen links abgeschlossen und rechts offen seien.
        \begin{solution}

            \qed
        \end{solution}
        
        \part 
        Bestimmen und skizzieren Sie die empirische Verteilungsfunktion aus den klassier-
        ten Daten.
        \begin{solution}

            \qed
        \end{solution}
    \end{parts}
\end{questions}
\end{document}