
\documentclass{abgabe}
\begin{document}

\begin{questions}
    \question
    Gegeben sind die folgenden Angaben zu den Fehlerzahlen in einer Statistik-Klausur:
    \begin{tabular}{c|c}
        Fehler & Anzahl \\ 
        \hline 
        0      & 18     \\ 
        1      & 22     \\ 
        2      & 15     \\ 
        3      & 11     \\ 
        4      & 8      \\ 
        5      & 4      \\ 
        6      & 2
    \end{tabular}
    \begin{parts}
        \part 
        Stellen Sie
        \begin{subparts}
            \subpart 
            die Wahrscheinlichkeitsfunktion und
            \begin{solution}

                \qed
            \end{solution}
            
            \subpart 
            Verteilungsfunktion
            \begin{solution}

                \qed
            \end{solution}
            graphisch dar.
        \end{subparts}
        
        \part 
        Berechnen Sie folgende Kenngrößen der Verteilung:
        \begin{subparts}
            \subpart 
            das arithmetische Mittel,
            \begin{solution}

                \qed
            \end{solution}
            
            \subpart 
            den Modalwert und
            \begin{solution}

                \qed
            \end{solution}
            
            \subpart 
            den Median.
            \begin{solution}

                \qed
            \end{solution}
        \end{subparts}
    \end{parts}
\end{questions}
\end{document}