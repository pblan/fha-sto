
\documentclass{abgabe}
\begin{document}

\begin{questions}
    \question
    Um die Qualität eines Zielfernrohres zu bewerten, lässt man 20 Schützen auf eine Zielscheibe schießen und registiert die Anzahl der Schüsse, bei denen ein Schütze das erste Mal die Mitte der Zielscheibe trifft. 
    Die Anzahl der Schüsse bis zum ersten Erfolg kann für alle Schützen als Zufallsvariable $X$ angesehen werden.
    \begin{parts}
        \part 
        Welche Verteilung liegt der Qualitätsüberprüfung zugrunde?
        \begin{solution}

            \qed
        \end{solution}
        
        \part 
        Bestimmen Sie auf Basis der folgenden Stichprobenergebnisse den Maximum-Likelihood-Schätzwert für den Parameter der Verteilung.
        \begin{center}
            \begin{tabular}{c|CCCC}
                $X = $ Anzahl der Fehlschüsse vor dem 1. Volltreffer & 1 & 2 & 3  & 4 \\ 
                Anzahl der Schützen                                  & 2 & 7 & 10 & 1
            \end{tabular}
        \end{center}
        \begin{solution}

            \qed
        \end{solution}
        
        \part 
        
        \begin{solution}

            \qed
        \end{solution}
    \end{parts}
\end{questions}
\end{document}