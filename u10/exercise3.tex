
\documentclass{abgabe}
\begin{document}

\begin{questions}
    \question
    Die Zufallsvariablen $X_1,\ldots ,X_n$ seien unabhängig und identisch verteilt mit der Dichte
    \[ 
        f_{\theta}(x) = 
        \begin{cases}
            e^{-(x-\theta)} & \text{für} \ x \geq \theta - 1 \\ 
            0               & \text{sonst}
        \end{cases}
        \quad (\theta > 0)
    \]
    \begin{parts}
        \part 
        Berechnen Sie einen Maximum-Likelihood-Schätzer für $\theta$.
        
        \emph{Hinweis:} Nicht alle Extremwerte findet man durch Differentiation \ldots
        \begin{solution}

            \qed
        \end{solution}
        
        \part 
        Gemessen wurden die folgenden 10 Werte:
        \[ 
            2,71 ; 2,43; 3,87; 4,12; 2,36; 2,24; 3,53; 3,28; 2,96; 2,87
        \]
        
        Berechnen Sie den Maximum-Likelihood-Schätzwert $\theta$ aus dieser Messreihe.
        \begin{solution}

            \qed
        \end{solution}
    \end{parts}
\end{questions}
\end{document}