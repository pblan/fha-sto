
\documentclass{abgabe}
\begin{document}

\begin{questions}
    \question
    Zur Erforschung der Erdkruste sollen Bohrungen in mehreren tausend Metern Tiefe durchgeführt werden. 
    Die tägliche Bohrleistung in $[m]$ eines dafür entwickelten Bohrgeräts wird als Zufallsvariable $X$ angesehen, wobei $X$ als gleichverteilt in einem Intervall $[0; b]$ mit unbekanntem $b$ angenommen wird. 
    Die bei Probebohrungen gemessenen täglichen Bohrleistungen werden als Realisierungen einer einfachen Stichprobe $X_1,\ldots,X_n$ aufgefasst.
    
    Zur Schätzung des Erwartungswerts $\mu = \Mean(X)$ wird die Schätzfunktion 
    \[ 
        \hat{\Theta}_1 = \conj{X}_n = \frac{1}{n} \sum^n_{i=1} X_i
    \]
    vorgeschlagen. Ist die Schätzfunktion
    \begin{parts}
        \part 
        erwartungstreu?
        \begin{solution}

            \qed
        \end{solution}
        
        \part 
        konsistent?
        \begin{solution}

            \qed
        \end{solution}
    \end{parts}
\end{questions}
\end{document}